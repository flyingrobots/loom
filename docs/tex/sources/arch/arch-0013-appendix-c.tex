\section{Deterministic Compliance Gauntlet}\label{sec:appendix-gauntlet}

This appendix defines a minimum compliance suite (the ``Gauntlet'') for
implementations of JS-ABI~v1.0.

An implementation \textbf{MUST} pass all tests in this appendix to be
considered JS-ABI~v1.0 compliant for the purposes of write-ahead
logging, hashing, and deterministic replay.

The Gauntlet is divided into:

\begin{itemize}[nosep]
  \item Encoder tests (CBOR encoder behaviour),
  \item Decoder tests (CBOR decoder behaviour),
  \item End-to-end tests (kernel logical clock and WAL behaviour).
\end{itemize}

Test vector identifiers (TV1, TV2, \dots) refer to the Reference Test
Vectors appendix.

\subsection{Encoder Compliance Tests}

\subsubsection*{EC-01: Exact Encoding of Handshake (TV1)}

Input: the logical handshake structure defined for TV1.

Requirement: when serialized under the JS-ABI~v1.0 deterministic profile,
the payload bytes \emph{must exactly match} the canonical CBOR hex for
TV1, and the header \texttt{LENGTH} field must be \texttt{0x00000071}
(113 decimal). Any deviation fails EC-01.

\subsubsection*{EC-02: Exact Encoding of Error (TV3)}

Input: the logical error structure defined for TV3.

Requirement: the encoded payload \emph{must exactly match} the canonical
CBOR hex for TV3, and \texttt{LENGTH} must be \texttt{0x00000076}
(118 decimal). Any deviation fails EC-02.

\subsubsection*{EC-03: Integer Minimal Width}

Input: a set of integer values (e.g., 0, 1, 10, 23, 24, 255, 256, 1000,
65535, 65536, \(2^{32}-1\)).

Requirement: each value \textbf{must} be encoded using the shortest
possible CBOR integer representation as per RFC~8949 Preferred
Serialization. Any larger-than-minimal integer encoding fails EC-03.

\subsubsection*{EC-04: Integer vs Float Encoding}

Input: numeric values \(0, 1, -1, 42, 0.0, 1.0, -1.0, 0.5, 1.5\).

Requirement:

\begin{itemize}[nosep]
  \item \(0, 1, -1, 42, 0.0, 1.0, -1.0\) must be encoded as integers
        (major type~0 or~1), not as floating-point.
  \item \(0.5\) and \(1.5\) must be encoded as floating-point using the
        smallest width that preserves the value exactly.
\end{itemize}

Emitting a floating-point where an integer is required, or using a
longer-than-necessary floating-point format, fails EC-04.

\subsubsection*{EC-05: Canonical Map Ordering}

Input: maps whose keys are chosen to exercise canonical ordering (e.g.,
including keys such as \texttt{\"op\"}, \texttt{\"ts\"}, \texttt{\"payload\"},
\texttt{\"a\"}, \texttt{\"aa\"}, \texttt{\"Z\"}).

Requirement: keys must be ordered by their full CBOR encoding (bytewise
increasing), matching the ordering implied by the canonical encodings in
TV1 and TV3. Any mismatch fails EC-05.

\subsubsection*{EC-06: No Tags, No Indefinite Length}

Input: any OpEnvelope payload.

Requirement: the encoder must not emit CBOR tags (major type~6),
indefinite-length strings, arrays, or maps, nor the break code
(\texttt{0xff}). Any such encoding fails EC-06.

\subsection{Decoder Compliance Tests}

\subsubsection*{DC-01: Accept Canonical TV1 and TV3}

Input: the exact CBOR payload bytes of TV1 and TV3.

Requirement: the decoder must successfully decode both payloads to the
expected logical structures. Failure to decode these encodings fails
DC-01.

\subsubsection*{DC-02: Reject Indefinite-Length Encodings}

Input: variants of TV1 in which one or more maps, arrays, or strings are
encoded using indefinite length with a break code, but with identical
logical content.

Requirement: the decoder must treat any use of indefinite-length
encodings as non-compliant and reject the packet as JS-ABI~v1.0
payload. Accepting such a payload fails DC-02.

\subsubsection*{DC-03: Reject Non-Canonical Integers}

Input: variants of TV1 in which an integer (such as \texttt{ts = 0} or
\texttt{client\_version = 1}) is encoded using a wider-than-necessary
integer format.

Requirement: the decoder may parse such payloads for diagnostics, but
must treat them as non-compliant with JS-ABI~v1.0 and must not accept
them into WAL or deterministic replay. Treating such a payload as normal
input fails DC-03.

\subsubsection*{DC-04: Reject Tags}

Input: variants of TV3 that introduce a CBOR tag around one of the
fields while preserving logical content.

Requirement: any CBOR tag in the payload must cause the decoder to
reject the packet as JS-ABI~v1.0 input. Accepting a tagged payload fails
DC-04.

\subsubsection*{DC-05: Reject Duplicate Map Keys}

Input: variants of TV1 or TV3 in which the top-level map or the
\texttt{\"payload\"} map contain duplicate keys.

Requirement: duplicate keys must be treated as a violation of the
deterministic encoding profile and cause the payload to be rejected.
Silently choosing one entry and continuing fails DC-05.

\subsubsection*{DC-06: Map Key Ordering Independence (Strict Mode)}

Input: variants of TV1 where map keys are emitted in non-canonical
order, while preserving logical fields.

Requirement: in strict JS-ABI~v1.0 mode, such non-canonical payloads
must be rejected for use in WAL and deterministic replay. Using them as
if they were canonical input fails DC-06. (Implementations may offer a
separate lenient mode for tooling, but that mode is outside JS-ABI~v1.0
compliance.)

\subsection{End-to-End Logical Clock and WAL Tests}

\subsubsection*{EE-01: Server-Assigned Timestamps}

Scenario:

\begin{enumerate}[nosep]
  \item Start with an empty kernel and WAL.
  \item A client sends two requests (e.g., a handshake and a
        state-mutating operation) with \texttt{ts = 0}.
  \item The server processes both and appends the resulting operations
        to the WAL.
\end{enumerate}

Requirement: the WAL entries must have strictly increasing \texttt{ts}
values, and the committed \texttt{ts} values must be assigned by the
kernel logical clock, not copied from the client. Any non-monotonic or
zero \texttt{ts} in the WAL fails EE-01.

\subsubsection*{EE-02: Monotonicity Under Concurrency}

Scenario:

\begin{enumerate}[nosep]
  \item Multiple clients concurrently issue state-mutating operations
        with arbitrary \texttt{ts} values (including 0 and stale values).
  \item The server interleaves processing and appends all committed
        operations to the WAL.
\end{enumerate}

Requirement: the sequence of WAL entries must have strictly increasing
\texttt{ts} values, forming a total order consistent with execution.
Any duplicate or non-monotonic \texttt{ts} values fail EE-02.

\subsubsection*{EE-03: Deterministic Replay}

Scenario:

\begin{enumerate}[nosep]
  \item From a known initial state, execute a sequence of valid JS-ABI
        operations, recording the resulting WAL and final state.
  \item Reinitialize the kernel to the same initial state.
  \item Replay the WAL entries in strictly increasing \texttt{ts} order,
        using the recorded payloads verbatim.
\end{enumerate}

Requirement: the resulting kernel state and any observable OpEnvelope
sequence must match the original. Failure to reproduce the same state or
sequence fails EE-03.

\subsection{Compliance Definition}

An implementation is considered JS-ABI~v1.0
\emph{deterministic-compliant} for WAL and replay if and only if:

\begin{itemize}[nosep]
  \item It passes all encoder tests EC-01 through EC-06,
  \item It passes all decoder tests DC-01 through DC-06, and
  \item It passes all end-to-end tests EE-01 through EE-03.
\end{itemize}

Failing any test means the implementation may still interoperate in a
best-effort fashion but cannot be trusted for cross-language
deterministic WAL, hashing, or replay, and must not be advertised as
JS-ABI~v1.0 compliant.
