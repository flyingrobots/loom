Time to manifest Section 6 --- one of the MOST important chapters in the
entire architecture doc, because this is where we explain:

\begin{itemize}
\tightlist
\item
  What memory means in a causal OS
\item
  Why JITOS does NOT use RAM-as-truth
\item
  How SWS overlays exist privately
\item
  How semantic memory behaves
\item
  Why ephemeral caches must evaporate
\item
  How global truth is accessed
\item
  Why the RMG is the only correct substrate
\item
  How zoom-level memory representations interact
\item
  How collapse interacts with memory
\item
  How agents (LLMs, tools, humans) see different memory layers
\item
  Why this is the world's first deterministic, replayable memory model
\end{itemize}

Let's carve it.

\begin{center}\rule{0.5\linewidth}{0.5pt}\end{center}

\section{JITOS Architecture Document --- Section
6}\label{jitos-architecture-document-section-6}

\begin{quote}
The Memory Model: Global Causality, Local Shadows, Semantic Depth
\end{quote}

(Grounded in ADR-0006, RFC-0019)

\begin{center}\rule{0.5\linewidth}{0.5pt}\end{center}

\subsection{1. The Memory Model}\label{the-memory-model}

\subsubsection{6.1 Overview}\label{overview}

Every operating system defines a memory model. In classical systems,
this means:

\begin{itemize}
\tightlist
\item
  RAM vs Disk
\item
  Heap vs Stack
\item
  Pages vs Frames
\item
  Addresses vs Values
\item
  Mutable cells
\item
  Aliasing and pointer graphs
\item
  The illusion of instantly shared state
\end{itemize}

These assumptions work only in:

\begin{itemize}
\tightlist
\item
  single-user machines
\item
  non-distributed computation
\item
  imperative languages
\item
  linear workflows
\item
  localized contexts
\end{itemize}

JITOS rejects all of these assumptions.

Instead, JITOS defines memory as:

A two-tier, multi-layered system combining an immutable global substrate
(RMG) with mutable, isolated local memory (SWS), enriched by structured
semantic graphs and supported by ephemeral compute caches.

This model is aligned with:

\begin{itemize}
\tightlist
\item
  causal determinism
\item
  multi-agent concurrency
\item
  semantic computing
\item
  reproducibility
\item
  distributed systems
\item
  formal reasoning
\item
  C\ensuremath{\Omega}MPUTER theory
\end{itemize}

It is the first memory model designed for the causal computing age.

\begin{center}\rule{0.5\linewidth}{0.5pt}\end{center}

\subsection{6.2 Memory Architecture
Overview}\label{memory-architecture-overview}

JITOS memory has four layers, grouped into two domains:

GLOBAL MEMORY (Immutable Reality)

Layer 0: RMG Substrate (Truth)

\begin{itemize}
\tightlist
\item
  immutable
\item
  append-only
\item
  multi-scale
\item
  causally ordered
\item
  infinite depth
\item
  shared by all
\item
  accessed via projection
\end{itemize}

This is the objective memory of the universe.

\begin{center}\rule{0.5\linewidth}{0.5pt}\end{center}

LOCAL MEMORY (Shadow / Subjective)

Layer 1: SWS Overlay Memory (Speculative State)

\begin{itemize}
\tightlist
\item
  mutable
\item
  private
\item
  isolated
\item
  structured as a graph
\item
  created on SWS creation
\item
  destroyed on collapse/discard
\end{itemize}

This is where computation happens.

\begin{center}\rule{0.5\linewidth}{0.5pt}\end{center}

Layer 2: Semantic Memory (Meaning)

\begin{itemize}
\tightlist
\item
  ASTs
\item
  semantic deltas
\item
  symbol tables
\item
  analysis results
\item
  LLM reasoning graphs
\item
  provenance annotations
\item
  structured transforms
\end{itemize}

This gives interpretation to computations.

\begin{center}\rule{0.5\linewidth}{0.5pt}\end{center}

Layer 3: Ephemeral Compute Memory (ECM)

\begin{itemize}
\tightlist
\item
  caches
\item
  build artifacts
\item
  lint results
\item
  intermediate IR
\item
  temporary storage
\end{itemize}

It is not part of truth, not preserved across SWS boundaries, and
evaporates afterwards.

\begin{center}\rule{0.5\linewidth}{0.5pt}\end{center}

\subsection{6.3 Layer 0: Global Memory (RMG
Substrate)}\label{layer-0-global-memory-rmg-substrate}

Global memory is represented by the Recursive Meta-Graph:

\begin{itemize}
\tightlist
\item
  all truth
\item
  all events
\item
  all structure
\item
  all history
\item
  all relationships
\item
  all semantic provenance
\item
  all identities
\item
  all universes
\end{itemize}

The RMG serves as:

\begin{itemize}
\tightlist
\item
  persistent memory
\item
  global state
\item
  root-of-truth
\item
  causal structure
\item
  audit trail
\item
  semantic knowledge base
\end{itemize}

Nothing in global memory ever mutates.

Global memory is identical on all machines after sync and replay.

\begin{center}\rule{0.5\linewidth}{0.5pt}\end{center}

\subsection{6.4 Layer 1: SWS Overlay Memory (Local
Speculation)}\label{layer-1-sws-overlay-memory-local-speculation}

Every SWS contains its own local memory:

\begin{itemize}
\tightlist
\item
  overlay nodes
\item
  local diffs
\item
  pending rewrites
\item
  uncommitted semantics
\item
  private states
\item
  merge candidates
\end{itemize}

Overlays are:

\begin{itemize}
\tightlist
\item
  mutable
\item
  safe
\item
  ephemeral
\item
  isolated
\end{itemize}

But they exist only within the SWS.

When collapse happens:

\begin{itemize}
\tightlist
\item
  overlays {\textrightarrow} RMG region
\item
  new snapshot is born
\item
  RMG expands
\item
  SWS dies
\item
  overlays are removed
\end{itemize}

This layer is the sandbox of computation.

\begin{center}\rule{0.5\linewidth}{0.5pt}\end{center}

\subsection{6.5 Layer 2: Semantic Memory
(Meaning)}\label{layer-2-semantic-memory-meaning}

Semantic memory exists inside the SWS, but is preserved into provenance
nodes upon collapse.

Semantic memory includes:

\begin{itemize}
\tightlist
\item
  abstract syntax trees
\item
  semantic deltas
\item
  LLM reasoning traces
\item
  symbolic analysis
\item
  transformation metadata
\item
  dependency graphs
\item
  type inference results
\item
  build graphs
\end{itemize}

This memory layer:

\begin{itemize}
\tightlist
\item
  is internal
\item
  is structured
\item
  is deeply nested
\item
  represents the ``why''
\item
  provides meaning to changes
\end{itemize}

And is fully compatible with RMG layering:

Semantic memory = RMG-in-RMG.

This allows:

\begin{itemize}
\tightlist
\item
  LLM autonomy
\item
  tool-based reasoning
\item
  semantic refactors
\item
  higher-order transforms
\end{itemize}

All in the same substrate.

\begin{center}\rule{0.5\linewidth}{0.5pt}\end{center}

\subsection{6.6 Layer 3: Ephemeral Compute Memory
(ECM)}\label{layer-3-ephemeral-compute-memory-ecm}

Temporary memory includes:

\begin{itemize}
\tightlist
\item
  build outputs
\item
  analysis intermediates
\item
  caches
\item
  partial diffs
\item
  scratch files
\end{itemize}

ECM is:

\begin{itemize}
\tightlist
\item
  not preserved
\item
  not exposed
\item
  not synced
\item
  not in the substrate
\item
  not part of SWS collapse
\end{itemize}

It is throwaway memory living only as long as necessary.

This prevents:

\begin{itemize}
\tightlist
\item
  RMG pollution
\item
  bloat
\item
  non-deterministic state history
\item
  unnecessary event nodes
\end{itemize}

\begin{center}\rule{0.5\linewidth}{0.5pt}\end{center}

\subsection{6.7 Memory Isolation and
Safety}\label{memory-isolation-and-safety}

JITOS's memory model enforces:

{\bfseries ?} No shared mutable state {\bfseries ?} No concurrent writes {\bfseries ?} No races {\bfseries ?} No
nondeterminism {\bfseries ?} No aliasing errors {\bfseries ?} No UAF, no double free, no
corruption {\bfseries ?} SWS isolation {\bfseries ?} RMG immutability {\bfseries ?} semantic separation

It is the safest, cleanest, most robust memory model ever designed.

\begin{center}\rule{0.5\linewidth}{0.5pt}\end{center}

\subsection{6.8 Memory and Collapse}\label{memory-and-collapse}

Collapse transforms memory:

Before collapse:

\begin{itemize}
\tightlist
\item
  overlays are mutable
\item
  semantic memory lives in SWS
\item
  ECM holds ephemeral data
\item
  SWS owns all speculation
\end{itemize}

During collapse:

\begin{itemize}
\tightlist
\item
  overlays {\textrightarrow} snapshot structure
\item
  semantic memory {\textrightarrow} provenance
\item
  ECM {\textrightarrow} evaporates
\item
  RMG expands
\item
  SWS ceases to exist
\end{itemize}

After collapse:

\begin{itemize}
\tightlist
\item
  RMG holds new truth
\item
  MH re-projects
\item
  SWS replaced by updated state
\end{itemize}

Collapse is the memory commit phase.

\begin{center}\rule{0.5\linewidth}{0.5pt}\end{center}

\subsection{6.9 Memory and Multi-Agent
Systems}\label{memory-and-multi-agent-systems}

Agents do not share SWS memory. They do not share ephemeral memory. They
do not share semantic memory.

They share only the RMG truth layer after collapse.

This ensures:

\begin{itemize}
\tightlist
\item
  no races
\item
  no interference
\item
  predictable behavior
\item
  safe parallelism
\item
  reproducible workflows
\item
  cooperation through collapse
\end{itemize}

JITOS is the first OS truly designed for agent-native computing.

\begin{center}\rule{0.5\linewidth}{0.5pt}\end{center}

\subsection{6.10 Summary}\label{summary}

JITOS defines memory not as:

\begin{itemize}
\tightlist
\item
  RAM
\item
  buffers
\item
  addresses
\item
  heaps
\item
  stacks
\end{itemize}

but as:

\begin{itemize}
\tightlist
\item
  Truth (RMG)
\item
  Speculation (SWS overlays)
\item
  Meaning (semantic memory)
\item
  Ephemera (ECM)
\end{itemize}

This model is:

\begin{itemize}
\tightlist
\item
  deterministic
\item
  reproducible
\item
  concurrent
\item
  semantic
\item
  causal
\item
  multi-layered
\item
  multi-agent safe
\end{itemize}

It is the memory model that classical computing should have invented
decades ago. Memory no longer means ``locations.'' Memory means
geometry.
