\section{\texorpdfstring{\textbf{Chapter 6 --- Worldlines: Execution as
Geodesics}}{Chapter 6 --- Worldlines: Execution as Geodesics}}\label{chapter-6-worldlines-execution-as-geodesics}

\subsection{\texorpdfstring{\textbf{What it means for a computation to
move.}}{What it means for a computation to move.}}\label{what-it-means-for-a-computation-to-move.}

Up to now, we've been talking about \textbf{possibility} --- the cone of
futures (Kairos), the structure that shapes it (Aios), and the path you
took to get here (Chronos).

Now we turn to the thing engineers care about most:

\begin{quote}
\textbf{What path does the computation ACTUALLY take?}

Not what it \emph{could} do. Not what it \emph{might} do. Not what's
adjacent in possibility space.

\textbf{What it DOES.}
\end{quote}

This path --- the one the system \emph{actually} walks --- is called a
\textbf{worldline}.

Worldlines are how your universe moves.

They are the ``film strip'' of your computation, one tick at a time, as
each DPO rewrite collapses the Time Cube into a single next step.

This chapter is about:

\begin{itemize}
\tightlist
\item
  how worldlines form,\\
\item
  why they're deterministic,
\item
  why they matter,
\item
  why they feel like ``program behavior,''
\item
  and why optimization \& debugging are both just geometry.
\end{itemize}

This is the chapter where execution stops being a mystery and becomes a
path through possibility space.

\begin{center}\rule{0.5\linewidth}{0.5pt}\end{center}

\section{\texorpdfstring{\textbf{6.1 --- What Is a
Worldline?}}{6.1 --- What Is a Worldline?}}\label{what-is-a-worldline}

A worldline is:

\begin{quote}
\textbf{A sequence of legal DPO rewrites} \textbf{applied to your RMG
universe} \textbf{under a deterministic scheduler.}
\end{quote}

In plain language:

\begin{itemize}
\tightlist
\item
  Each tick: one rewrite fires\\
\item
  The rewrite transforms the RMG
\item
  The new RMG is the new universe state
\item
  Repeat
\end{itemize}

That chain of states, from tick 0 {\textrightarrow} tick N, is your \textbf{worldline}.
This is not a metaphor. This is literally what execution \emph{is} in an
RMG runtime.

Not ``running code.'' Not ``executing instructions.''

Just:

\textbf{State {\textrightarrow} rewrite {\textrightarrow} state {\textrightarrow} rewrite {\textrightarrow} state}

A worldline is the \emph{actual} history.

\begin{center}\rule{0.5\linewidth}{0.5pt}\end{center}

\section{\texorpdfstring{\textbf{6.2 --- Why C\ensuremath{\Omega}MPUTER's Worldlines Are
Deterministic}}{6.2 --- Why C\ensuremath{\Omega}MPUTER's Worldlines Are Deterministic}}\label{why-cux3c9mputers-worldlines-are-deterministic}

Raw DPO is nondeterministic.

Classical rewrite systems have no opinion about which rule fires first.

But in C\ensuremath{\Omega}MPUTER, we introduce:

\begin{itemize}
\tightlist
\item
  a scheduler\\
\item
  a tick
\item
  priority rules
\item
  canonical match order
\item
  deterministic tie-breaking
\item
  conflict resolution
\item
  no-overlap constraints
\item
  explicit observer semantics
\end{itemize}

And with that:

\begin{quote}
\textbf{Every worldline becomes deterministic.}
\end{quote}

One tick {\textrightarrow} one rewrite {\textrightarrow} one next universe. No randomness. No
nondeterminism. No ambiguity.

This is what makes analysis possible. It lets debugging become
deterministic archaeology, and optimization become deterministic
navigation.

The observer (scheduler) isn't magical. It's just the thing that picks
one legal path out of the Time Cube. Like choosing one door in the room.

\begin{center}\rule{0.5\linewidth}{0.5pt}\end{center}

\section{\texorpdfstring{\textbf{6.3 --- Worldline Sharpness: Why Small
Changes
Matter}}{6.3 --- Worldline Sharpness: Why Small Changes Matter}}\label{worldline-sharpness-why-small-changes-matter}

Imagine two worldlines that share the same first 100 ticks and then
diverge at tick 101. From that point on, they become different
universes. Maybe similar at first. But differences compound. A small
change in a rewrite 3 layers deep inside a wormhole can have large
effects later.

This is \textbf{worldline sharpness}:

\begin{quote}
\textbf{The sensitivity of a universe to small differences in its
rewrite history.}
\end{quote}

Not chaos theory. Not randomness. Just structure.

Systems with low curvature (Chapter 9) tend to have soft, flexible
worldlines --- small changes don't derail everything.

Systems with high curvature tend to have brittle worldlines --- tiny
changes break everything.

Every engineer has felt this. Now you have the language to describe it.

\begin{center}\rule{0.5\linewidth}{0.5pt}\end{center}

\section{\texorpdfstring{\textbf{6.4 --- Geodesics: The ``Straight
Lines'' of
Computation}}{6.4 --- Geodesics: The ``Straight Lines'' of Computation}}\label{geodesics-the-straight-lines-of-computation}

Once we define Rulial Distance (Chapter 5), we can ask the question:

\begin{quote}
\textbf{What is the shortest path from the initial state to the final
state?}
\end{quote}

This path --- the minimal rewrite path --- is the computational
\textbf{geodesic}.

In a perfect world:

\begin{itemize}
\tightlist
\item
  your optimized program follows a geodesic,\\
\item
  your debugged program restores the geodesic,
\item
  your refactoring straightens the geodesic,
\item
  your compiler finds shorter geodesics automatically.
\end{itemize}

In real terms:

\begin{itemize}
\tightlist
\item
  fewer rewrites\\
\item
  simpler transformations
\item
  lower cost
\item
  fewer steps
\item
  less branching
\item
  more direct worldline
\end{itemize}

Optimization stops being black magic. It becomes a geometric process:

\begin{quote}
\textbf{Make the worldline straighter.}
\end{quote}

\begin{center}\rule{0.5\linewidth}{0.5pt}\end{center}

\section{\texorpdfstring{\textbf{6.5 --- Collapse: Choosing One
Future}}{6.5 --- Collapse: Choosing One Future}}\label{collapse-choosing-one-future}

The Time Cube gives you a cone of futures. The scheduler picks one.

This is \textbf{collapse}.

It's not quantum. It's not random. It's not metaphysical.

Collapse is the moment when:

\begin{itemize}
\tightlist
\item
  you match L\\
\item
  validate K
\item
  apply R
\item
  commit the rewrite
\item
  and advance Chronos by one tick
\end{itemize}

Collapse shrinks the Time Cube into a single next tick and produces the
next RMG universe. This is \textbf{control flow} in RMG terms.

Every collapse is:

\begin{itemize}
\tightlist
\item
  a choice\\
\item
  a commitment
\item
  a reduction in possibility
\item
  a step deeper into your worldline
\end{itemize}

\begin{center}\rule{0.5\linewidth}{0.5pt}\end{center}

\section{\texorpdfstring{\textbf{6.6 --- Worldlines Are
Debugging}}{6.6 --- Worldlines Are Debugging}}\label{worldlines-are-debugging}

Debugging in RMG terms is simple:

\begin{quote}
\textbf{A worldline didn't go where you wanted.} \textbf{Trace it back.}
\end{quote}

You're not inspecting ``stack traces'' or ``AST nodes'' or ``function
calls.''

You're inspecting:

\begin{itemize}
\tightlist
\item
  which wormhole was chosen at each tick
\item
  why it was legal
\item
  why others weren't
\item
  how the universe changed
\item
  how Chronos diverged from the ideal geodesic
\end{itemize}

Debugging becomes archaeology:

\begin{quote}
The study of a computational past.
\end{quote}

And because RMG stores structure, you can \textbf{diff worldlines} and
measure how far apart execution paths really are in Rulial Distance.

That's not mystical --- it's just structural comparison.

\begin{center}\rule{0.5\linewidth}{0.5pt}\end{center}

\section{\texorpdfstring{\textbf{6.7 --- Worldlines Are
Optimization}}{6.7 --- Worldlines Are Optimization}}\label{worldlines-are-optimization}

Optimizing a system becomes:

\begin{quote}
\textbf{Find a shorter or straighter worldline} \textbf{from A to B.}
\end{quote}

This reframes:

\begin{itemize}
\tightlist
\item
  constant folding
\item
  dead code elimination
\item
  inline substitution
\item
  strength reduction
\item
  normalization
\item
  algebraic simplification
\item
  caching
\item
  memoization
\item
  JIT optimization
\end{itemize}

\ldots as geometric moves.

Optimization becomes:

\begin{itemize}
\tightlist
\item
  minimizing curvature\\
\item
  eliminating detours
\item
  reducing Rulial Distance
\item
  straightening paths
\item
  avoiding brittle regions
\item
  aligning chronos with geodesics
\end{itemize}

This is the hidden geometry behind why ``fast code feels elegant.''

\begin{center}\rule{0.5\linewidth}{0.5pt}\end{center}

\section{\texorpdfstring{\textbf{FOR THE
NERDS{\texttrademark}}}{FOR THE NERDS{\texttrademark}}}\label{for-the-nerds}

\subsection{\texorpdfstring{\textbf{Worldlines and Lambda Calculus
Reduction
Sequences}}{Worldlines and Lambda Calculus Reduction Sequences}}\label{worldlines-and-lambda-calculus-reduction-sequences}

If you squint, a worldline is:

\begin{itemize}
\tightlist
\item
  a \ensuremath{\beta}-reduction trace,\\
\item
  a sequence of graph reductions,
\item
  a normal form search,
\item
  a deterministic rewrite strategy (call-by-X),
\item
  a reduction semantics with a fixed evaluation order.
\end{itemize}

But RMG+DPO worldlines:

\begin{itemize}
\tightlist
\item
  are multi-scale
\item
  include recursive edges
\item
  reflect wormhole structure
\item
  aren't term-based
\item
  aren't flat
\item
  include storage\\
\item
  include branching alternatives
\item
  include a geometry
\item
  are defined over typed transforms
\item
  and exist in a metric space
\end{itemize}

So the analogy holds --- but the structure is richer.

\emph{(End sidebar.)}

\begin{center}\rule{0.5\linewidth}{0.5pt}\end{center}

\section{\texorpdfstring{\textbf{6.8 --- Transition: From Worldlines to
Neighborhoods}}{6.8 --- Transition: From Worldlines to Neighborhoods}}\label{transition-from-worldlines-to-neighborhoods}

We know:

\begin{itemize}
\tightlist
\item
  what possibility looks like (Time Cube),\\
\item
  how paths form (worldlines),
\item
  how geometry governs those paths (distance, geodesics),
\item
  and how determinism collapses possibility into history.
\end{itemize}

Now we need to understand:

\begin{quote}
\textbf{What does the area around a worldline look like?}
\end{quote}

\begin{itemize}
\tightlist
\item
  How do worlds cluster?\\
\item
  Why are some universes adjacent?
\item
  Why are others ``far'' in possibility space?
\item
  Why do some futures feel ``available'' and others don't?
\item
  How does structure shape neighborhoods?
\end{itemize}

This is the domain of Chapter 7. Where we study the \textbf{local
geometry} around a worldline --- the neighborhoods that define the feel
of a system.

\begin{center}\rule{0.5\linewidth}{0.5pt}\end{center}

\section{\texorpdfstring{\textbf{C\ensuremath{\Omega}MPUTER {\textbullet}
JITOS}}{C\ensuremath{\Omega}MPUTER {\textbullet} JITOS}}\label{cux3c9mputer-jitos}

{\textcopyright} 2025 James Ross {\textbullet} \href{https://flyingrobots.dev}{Flying {\textbullet} Robots} All
Rights Reserved
