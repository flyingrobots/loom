\section{\texorpdfstring{\textbf{Chapter 8 --- MRMW: The Phase Space of
All Possible
Computations}}{Chapter 8 --- MRMW: The Phase Space of All Possible Computations}}\label{chapter-8-mrmw-the-phase-space-of-all-possible-computations}

\subsection{\texorpdfstring{\textbf{The cosmology of one computational
universe.}}{The cosmology of one computational universe.}}\label{the-cosmology-of-one-computational-universe.}

Up to now, we've explored \emph{your} universe:

\begin{itemize}
\tightlist
\item
  your RMG structure,\\
\item
  your DPO rules,
\item
  your worldline,
\item
  your Time Cube,
\item
  your neighborhoods.
\end{itemize}

Part II so far has lived at the \textbf{local} scale --- the ``here'' of
your computational reality. But every universe is defined not just by
where it is, but by what's around it.

Today we zoom out one more level. Not infinitely --- not cosmically ---
just enough to see:

\begin{quote}
\textbf{your universe among its neighbors in rule-space.}
\end{quote}

Where Part I gave us the substrate and Part II gave us geometry, Chapter
8 gives us \textbf{cosmology}.

Not the cosmology of physics. The cosmology of computation --- the
structure of all possible \emph{models} and all possible
\emph{histories} you can generate with the same machinery.

Welcome to \textbf{MRMW}.

\begin{center}\rule{0.5\linewidth}{0.5pt}\end{center}

\section{\texorpdfstring{\textbf{8.1 --- Multiple Rulial Models (MR):
Rule-Space as a
Landscape}}{8.1 --- Multiple Rulial Models (MR): Rule-Space as a Landscape}}\label{multiple-rulial-models-mr-rule-space-as-a-landscape}

Every computational universe is defined by:

\begin{itemize}
\tightlist
\item
  a set of DPO rules,\\
\item
  their types (K-interfaces),
\item
  their rewritable regions,
\item
  their invariants,
\item
  and their scheduler semantics.
\end{itemize}

Change the rules, and you create a new universe. Not metaphorically.
Literally.

Small differences in:

\begin{itemize}
\tightlist
\item
  K-interfaces,\\
\item
  rewrite patterns,
\item
  constraints,
\item
  orderings,
\item
  or priorities
\end{itemize}

create entirely different behaviors, shapes, and worldlines.

Call each rule-system \textbf{a Rulial Model}.

Now imagine mapping the adjacency of these models:

\begin{itemize}
\tightlist
\item
  two models are ``near'' if their rules differ slightly,
\item
  far if their rules differ drastically,
\item
  smooth if changes preserve structure,\\
\item
  jagged if small changes break everything;
\end{itemize}

This produces a \textbf{rule-space} --- a \emph{landscape} of
computational universes.

Each one is an RMG+DPO cosmos. Each one is a way the world \emph{could}
behave if its laws were different.

This is MR: \textbf{Multiple Rulial Models.}

\begin{center}\rule{0.5\linewidth}{0.5pt}\end{center}

\section{\texorpdfstring{\textbf{8.2 --- Multiple Worldlines (MW):
Histories Within a
Model}}{8.2 --- Multiple Worldlines (MW): Histories Within a Model}}\label{multiple-worldlines-mw-histories-within-a-model}

Inside a single model, you have:

\begin{itemize}
\tightlist
\item
  one worldline (the one you took),\\
\item
  many possible worldlines (the ones you didn't),
\item
  and countless alternative futures (the Time Cube).
\end{itemize}

That cluster of worldlines --- all inside the same rule-system --- forms
\textbf{MW: Multiple Worldlines}.

So now you have two spaces:

\begin{enumerate}
\def\labelenumi{\arabic{enumi}.}
\tightlist
\item
  \textbf{MR} --- all \emph{rule variations}\\
\item
  \textbf{MW} --- all \emph{worldline variations} inside each model
\end{enumerate}

These two dimensions together form your computational phase space.

\begin{center}\rule{0.5\linewidth}{0.5pt}\end{center}

\section{\texorpdfstring{\textbf{8.3 --- MRMW: The Full Phase
Space}}{8.3 --- MRMW: The Full Phase Space}}\label{mrmw-the-full-phase-space}

Put MR and MW together and you get:

\begin{lstlisting}
MR x MW
\end{lstlisting}

The Rulial Phase Space of your computational universe.

This is where cosmology meets geometry. This is where universes touch at
the edges. This is where:

\begin{itemize}
\tightlist
\item
  small rule changes create new universes,
\item
  small worldline deviations create new histories,
\item
  neighborhoods appear in rule-space
\item
  \emph{and} execution-space simultaneously.
\end{itemize}

\subsubsection{\texorpdfstring{\textbf{The Multiverse isn't all
universes. It's all universes reachable by structured changes to rules
and to
decisions.}}{The Multiverse isn't all universes. It's all universes reachable by structured changes to rules and to decisions.}}\label{the-multiverse-isnt-all-universes.-its-all-universes-reachable-by-structured-changes-to-rules-and-to-decisions.}

This space is:

\begin{itemize}
\tightlist
\item
  finite-per-rule-set,\\
\item
  bounded,
\item
  computable,
\item
  structured,
\item
  navigable.
\end{itemize}

In other words: \textbf{useful.}

We're not exploring impossibilities. We're exploring relatives.

\begin{center}\rule{0.5\linewidth}{0.5pt}\end{center}

\section{\texorpdfstring{\textbf{8.4 --- The Time Cube Across
Models}}{8.4 --- The Time Cube Across Models}}\label{the-time-cube-across-models}

The coolest part of MRMW isn't what happens inside a model --- it's what
happens when you shift models slightly.

Each model has its own:

\begin{itemize}
\tightlist
\item
  Time Cube,\\
\item
  neighborhoods,
\item
  curvature,
\item
  adjacency,
\item
  and reachable worldlines.
\end{itemize}

Changing the rule-set changes:

\begin{itemize}
\tightlist
\item
  the shape of the cone,\\
\item
  the width of possibility,
\item
  the number of legal rewrites,
\item
  the landscape of adjacency,
\item
  the smoothness of optimization.
\end{itemize}

Some models have huge, smooth cones. Some models have tight, brittle
cones. Some have beautiful geodesic paths. Some have jagged labyrinths.

MRMW is where you can:

\begin{itemize}
\tightlist
\item
  compare universes,\\
\item
  measure robustness,
\item
  analyze rule sensitivity,
\item
  explore alternative semantics.
\end{itemize}

This is more than debugging. More than optimization. More than analysis.

This is \textbf{structural exploration.}

\begin{center}\rule{0.5\linewidth}{0.5pt}\end{center}

\section{\texorpdfstring{\textbf{8.5 --- Applications: Why MRMW
Matters}}{8.5 --- Applications: Why MRMW Matters}}\label{applications-why-mrmw-matters}

MRMW is not philosophy. It is a practical toolkit for:

\subsubsection{\texorpdfstring{\textbf{Debugging}}{Debugging}}\label{debugging}

Finding universes where invariants break.

\subsubsection{\texorpdfstring{\textbf{Optimization}}{Optimization}}\label{optimization}

Finding universes where worldlines are shorter.

\subsubsection{\texorpdfstring{\textbf{Design}}{Design}}\label{design}

Comparing rule-sets to find robust or expressive ones.

\subsubsection{\texorpdfstring{\textbf{AI
Reasoning}}{AI Reasoning}}\label{ai-reasoning}

Exploring alternative consistent computation paths.

\subsubsection{\texorpdfstring{\textbf{Language
Semantics}}{Language Semantics}}\label{language-semantics}

Viewing language design as model selection.

\subsubsection{\texorpdfstring{\textbf{Compilation}}{Compilation}}\label{compilation}

Searching for optimal rewrite sequences across models.

\subsubsection{\texorpdfstring{\textbf{Simulation}}{Simulation}}\label{simulation}

Finding universes with similar behavior under neighboring laws.

\subsubsection{\texorpdfstring{\textbf{Robustness
Analysis}}{Robustness Analysis}}\label{robustness-analysis}

Seeking universes stable under small rule perturbations.

MRMW turns design into geometry and geometry into engineering leverage.

\begin{center}\rule{0.5\linewidth}{0.5pt}\end{center}

\section{\texorpdfstring{\textbf{FOR THE
NERDS{\texttrademark}}}{FOR THE NERDS{\texttrademark}}}\label{for-the-nerds}

\subsection{\texorpdfstring{\textbf{MRMW as a Fiber Bundle Over
Rule-Space}}{MRMW as a Fiber Bundle Over Rule-Space}}\label{mrmw-as-a-fiber-bundle-over-rule-space}

If you know differential geometry:

MR (rule-space) is the base manifold. MW (worldlines) are the fibers.

MRMW is the full bundle.

But you don't need fiber bundles to use it.

Just know:

\begin{quote}
Changing rules changes the shape of possibility.

Changing worldlines changes the path through possibility.

MRMW is the space of all such changes.
\end{quote}

\emph{(End sidebar.)}

\begin{center}\rule{0.5\linewidth}{0.5pt}\end{center}

\section{\texorpdfstring{\textbf{8.6 --- Transition: The Sky
Opens}}{8.6 --- Transition: The Sky Opens}}\label{transition-the-sky-opens}

With Chapter 8, Part II completes its arc. You began in the cave with
structure. You stepped outside to see:

\begin{itemize}
\tightlist
\item
  possibility (Kairos),
\item
  path (Chronos),
\item
  arena (Aios),
\item
  adjacency,
\item
  curvature,
\item
  distance,
\item
  alternative worlds,
\item
  and whole universes formed by changing the rules.
\end{itemize}

This is the sky.

This is the shape of computation itself when you stop thinking in code
and start thinking in geometry.

Now you have the full lens:

\begin{itemize}
\tightlist
\item
  Worldlines\\
\item
  Time Cubes
\item
  Rulial Distance
\item
  Neighborhoods
\item
  MRMW
\end{itemize}

With these tools, you can finally understand the \emph{physics} of
computation.

\textbf{\emph{Not actual physics.}}

Just the laws that govern motion in RMG space. That's Part III.

\begin{center}\rule{0.5\linewidth}{0.5pt}\end{center}

\section{\texorpdfstring{\textbf{C\ensuremath{\Omega}MPUTER {\textbullet}
JITOS}}{C\ensuremath{\Omega}MPUTER {\textbullet} JITOS}}\label{cux3c9mputer-jitos}

{\textcopyright} 2025 James Ross {\textbullet} \href{https://flyingrobots.dev}{Flying {\textbullet} Robots} All
Rights Reserved
