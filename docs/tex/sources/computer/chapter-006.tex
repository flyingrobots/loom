\section{\textbf{Chapter 6 --- Worldlines: Execution as Geodesics}}

\subsection{\textbf{What it means for a computation to move.}}

Up to now, we  ve been talking about \textbf{possibility} --- the cone of futures (Kairos), the structure that shapes it (Aios), and the path you took to get here (Chronos).

Now we turn to the thing engineers care about most:

\begin{quote}
\itshape
\textbf{What path does the computation ACTUALLY take?}

Not what it \textit{could} do.
Not what it \textit{might} do.
Not what  s adjacent in possibility space.

\textbf{What it DOES.}
\end{quote}

This path --- the one the system \textit{actually} walks --- is called a \textbf{worldline}.

Worldlines are how your universe moves.

They are the    film strip   of your computation, one tick at a time, as each DPO rewrite collapses the Time Cube into a single next step.

This chapter is about:

\begin{itemize}
\item how worldlines form,
\item why they  re deterministic,
\item why they matter,
\item why they feel like    program behavior,  
\item and why optimization \& debugging are both just geometry.
\end{itemize}

This is the chapter where execution stops being a mystery and becomes a path through possibility space.

---

\section{\textbf{6.1 --- What Is a Worldline?}}

A worldline is:

\begin{quote}
\itshape
\textbf{A sequence of legal DPO rewrites}
\textbf{applied to your RMG universe}
\textbf{under a deterministic scheduler.}
\end{quote}

In plain language:

\begin{itemize}
\item Each tick: one rewrite fires
\item The rewrite transforms the RMG
\item The new RMG is the new universe state
\item Repeat
\end{itemize}

That chain of states, from tick 0 $\rightarrow$ tick N, is your \textbf{worldline}. This is not a metaphor. This is literally what execution \textit{is} in an RMG runtime.

Not    running code.   Not    executing instructions.  

Just:

\textbf{State $\rightarrow$ rewrite $\rightarrow$ state $\rightarrow$ rewrite $\rightarrow$ state}

A worldline is the \textit{actual} history.

---

\section{\textbf{6.2 --- Why C$\\Omega$MPUTER  s Worldlines Are Deterministic}}

Raw DPO is nondeterministic.

Classical rewrite systems have no opinion about which rule fires first.

  But in C$\\Omega$MPUTER, we introduce:

\begin{itemize}
\item a scheduler
\item a tick
\item priority rules
\item canonical match order
\item deterministic tie-breaking
\item conflict resolution
\item no-overlap constraints
\item explicit observer semantics
\end{itemize}

And with that:

\begin{quote}
\itshape
\textbf{Every worldline becomes deterministic.}
\end{quote}

One tick $\rightarrow$ one rewrite $\rightarrow$ one next universe.
No randomness.
No nondeterminism.
No ambiguity.

This is what makes analysis possible. It lets debugging become deterministic archaeology, and optimization become deterministic navigation.

The observer (scheduler) isn  t magical. It  s just the thing that picks one legal path out of the Time Cube. Like choosing one door in the room.

---

\section{\textbf{6.3 --- Worldline Sharpness: Why Small Changes Matter}}

Imagine two worldlines that share the same first 100 ticks and then diverge at tick 101. From that point on, they become different universes. Maybe similar at first. But differences compound. A small change in a rewrite 3 layers deep inside a wormhole can have large effects later.

This is \textbf{worldline sharpness}:

\begin{quote}
\itshape
\textbf{The sensitivity of a universe to small differences in its rewrite history.}
\end{quote}

Not chaos theory.
Not randomness.
Just structure.

Systems with low curvature (Chapter 9) tend to have soft, flexible worldlines --- small changes don  t derail everything.

Systems with high curvature tend to have brittle worldlines --- tiny changes break everything.

Every engineer has felt this. Now you have the language to describe it.

---

\section{\textbf{6.4 --- Geodesics: The    Straight Lines   of Computation}}

Once we define Rulial Distance (Chapter 5), we can ask the question:

\begin{quote}
\itshape
\textbf{What is the shortest path from the initial state to the final state?}
\end{quote}

This path --- the minimal rewrite path --- is the computational \textbf{geodesic}.

In a perfect world:

\begin{itemize}
\item your optimized program follows a geodesic,
\item your debugged program restores the geodesic,
\item your refactoring straightens the geodesic,
\item your compiler finds shorter geodesics automatically.
\end{itemize}

In real terms:

\begin{itemize}
\item fewer rewrites
\item simpler transformations
\item lower cost
\item fewer steps
\item less branching
\item more direct worldline
\end{itemize}

Optimization stops being black magic. It becomes a geometric process:

\begin{quote}
\itshape
\textbf{Make the worldline straighter.}
\end{quote}

---

\section{\textbf{6.5 --- Collapse: Choosing One Future}}

The Time Cube gives you a cone of futures. The scheduler picks one.

This is \textbf{collapse}.

It  s not quantum.
It  s not random.
It  s not metaphysical.

Collapse is the moment when:

\begin{itemize}
\item you match L
\item validate K
\item apply R
\item commit the rewrite
\item and advance Chronos by one tick
\end{itemize}

Collapse shrinks the Time Cube into a single next tick and produces the next RMG universe. This is \textbf{control flow} in RMG terms.

Every collapse is:

\begin{itemize}
\item a choice
\item a commitment
\item a reduction in possibility
\item a step deeper into your worldline
\end{itemize}

---

\section{\textbf{6.6 --- Worldlines Are Debugging}}

Debugging in RMG terms is simple:

\begin{quote}
\itshape
\textbf{A worldline didn  t go where you wanted.}
\textbf{Trace it back.}
\end{quote}

You  re not inspecting    stack traces   or    AST nodes   or    function calls.  

You  re inspecting:

\begin{itemize}
\item which wormhole was chosen at each tick
\item why it was legal
\item why others weren  t
\item how the universe changed
\item how Chronos diverged from the ideal geodesic
\end{itemize}

Debugging becomes archaeology:

\begin{quote}
\itshape
The study of a computational past.
\end{quote}

And because RMG stores structure, you can \textbf{diff worldlines} and measure how far apart execution paths really are in Rulial Distance.

That  s not mystical --- it  s just structural comparison.

---

\section{\textbf{6.7 --- Worldlines Are Optimization}}

Optimizing a system becomes:

\begin{quote}
\itshape
\textbf{Find a shorter or straighter worldline}
\textbf{from A to B.}
\end{quote}

This reframes:

\begin{itemize}
\item constant folding
\item dead code elimination
\item inline substitution
\item strength reduction
\item normalization
\item algebraic simplification
\item caching
\item memoization
\item JIT optimization
\end{itemize}


  as geometric moves.

Optimization becomes:

\begin{itemize}
\item minimizing curvature
\item eliminating detours
\item reducing Rulial Distance
\item straightening paths
\item avoiding brittle regions
\item aligning chronos with geodesics
\end{itemize}

This is the hidden geometry behind why    fast code feels elegant.  

---

\section{\textbf{FOR THE NERDS }}

\subsection{\textbf{Worldlines and Lambda Calculus Reduction Sequences}}

If you squint, a worldline is:

\begin{itemize}
\item a $\beta$-reduction trace,
\item a sequence of graph reductions,
\item a normal form search,
\item a deterministic rewrite strategy (call-by-X),
\item a reduction semantics with a fixed evaluation order.
\end{itemize}

But RMG+DPO worldlines:

\begin{itemize}
\item are multi-scale
\item include recursive edges
\item reflect wormhole structure
\item aren  t term-based
\item aren  t flat
\item include storage
\item include branching alternatives
\item include a geometry
\item are defined over typed transforms
\item and exist in a metric space
\end{itemize}

So the analogy holds --- but the structure is richer.

\textit{(End sidebar.)}

---

\section{\textbf{6.8 --- Transition: From Worldlines to Neighborhoods}}

We know:

\begin{itemize}
\item what possibility looks like (Time Cube),
\item how paths form (worldlines),
\item how geometry governs those paths (distance, geodesics),
\item and how determinism collapses possibility into history.
\end{itemize}

Now we need to understand:

\begin{quote}
\itshape
\textbf{What does the area around a worldline look like?}
\end{quote}

\begin{itemize}
\item How do worlds cluster?
\item Why are some universes adjacent?
\item Why are others    far   in possibility space?
\item Why do some futures feel    available   and others don  t?
\item How does structure shape neighborhoods?
\end{itemize}

This is the domain of Chapter 7. Where we study the \textbf{local geometry} around a worldline --- the neighborhoods that define the feel of a system.

---

\section{\textbf{C$\\Omega$MPUTER \textbullet{} JITOS}}
