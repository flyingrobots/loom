\section{ADR-0003 --- The Substrate Is a Recursive Meta-Graph
(RMG)}\label{adr-0003-the-substrate-is-a-recursive-meta-graph-rmg}

\begin{quote}
``We said DAG, but we meant RMG.''
\end{quote}

\subsection{1. Context}\label{context}

In early drafts and RFCs, the substrate of JITOS was described as:

\begin{itemize}
\tightlist
\item
  ``a causal DAG''
\item
  ``append-only event graph''
\item
  ``Git-like commit DAG''
\end{itemize}

This was correct only in the lowest layer of the reality stack.

But as the system evolved, especially through:

\begin{itemize}
\tightlist
\item
  SWS overlays
\item
  inversion rewrite nodes
\item
  provenance nodes
\item
  semantic attachments
\item
  file-chunk graphs
\item
  multi-agent overlays
\item
  federation edges
\item
  schema evolution
\item
  the C\ensuremath{\Omega}MPUTER fusion layer
\item
  and the need for multi-scale event representation (ADR-0004)
\end{itemize}

\ldots it became clear that the DAG model is insufficient to describe
the full substrate.

A pure DAG cannot:

\begin{itemize}
\tightlist
\item
  store graphs as node payloads
\item
  express multi-level structure
\item
  represent ASTs
\item
  represent semantic provenance graphs
\item
  store internal micro-events
\item
  represent multi-agent reasoning traces
\item
  support schema evolution as graph-of-graph rewriting
\item
  unify micro/macro commit semantics
\item
  serve as a foundation for a multi-universe federation
\item
  integrate with the C\ensuremath{\Omega}MPUTER metaphysics
\item
  maintain reflexive, semantic meta-structure
\end{itemize}

The DAG is only the first-order projection of a much richer structure.

Thus:

\begin{quote}
The substrate is not a DAG. The substrate is an RMG: a Recursive
Meta-Graph.
\end{quote}

\begin{center}\rule{0.5\linewidth}{0.5pt}\end{center}

\subsection{2. Decision}\label{decision}

JITOS's substrate is a Recursive Meta-Graph (RMG). The causal DAG is one
layer of this RMG, not the substrate itself.

This decision establishes:

\begin{itemize}
\tightlist
\item
  Node payloads MAY themselves be graphs
\item
  Node types are defined by meta-graphs
\item
  Schema evolution is rewriting the meta-graph
\item
  Provenance nodes attach subgraphs
\item
  Multi-agent overlays form local meta-graphs
\item
  Semantic representations (ASTs, IR, reasoning traces) are RMG layers
\item
  RMG supports multi-scale event representation
  (\href{./ADR-0004.md}{ADR-0004})
\item
  Federation is RMG-to-RMG linking
\item
  The kernel operates over structured, fractal, multi-layer graphs
\item
  The substrate is infinite in depth, not just length
\end{itemize}

The RMG is the true form of JITOS reality.

The DAG is a shadow of it --- a human-interpretable, flattened timeline
of causal events.

In other words:

\begin{quote}
\textbf{The DAG is to the RMG what the filesystem is to Materialized
Head. A projection, not the reality.}
\end{quote}

\begin{center}\rule{0.5\linewidth}{0.5pt}\end{center}

\subsection{3. Rationale}\label{rationale}

\subsubsection{3.1 RMG reflects the natural structure of
computation}\label{rmg-reflects-the-natural-structure-of-computation}

Code, data, semantics, transformations, provenance, and thought
processes are not linearly structured.

They are:

\begin{itemize}
\tightlist
\item
  nested
\item
  recursive
\item
  semantic
\item
  multi-level
\item
  graph-structured
\item
  fractal
\item
  self-referential
\end{itemize}

The RMG captures this truth.

\begin{center}\rule{0.5\linewidth}{0.5pt}\end{center}

\subsubsection{3.2 DAG-only cannot represent semantics or
structure}\label{dag-only-cannot-represent-semantics-or-structure}

A simple DAG fails to express:

\begin{itemize}
\tightlist
\item
  ASTs
\item
  reasoning trees
\item
  semantic diffs
\item
  provenance graphs
\item
  agent plans
\item
  build pipelines
\item
  dependency graphs
\item
  DPO rewrites
\end{itemize}

RMG allows all of these as first-class citizens.

\begin{center}\rule{0.5\linewidth}{0.5pt}\end{center}

\subsubsection{3.3 RMG enables multi-scale event representation
(ADR-0004)}\label{rmg-enables-multi-scale-event-representation-adr-0004}

Humans see:

\begin{itemize}
\tightlist
\item
  1 conceptual change
\end{itemize}

Machines see:

\begin{itemize}
\tightlist
\item
  30 keystrokes
\item
  12 semantic rewrites
\item
  1 macro refactor
\item
  1 merge resolution
\item
  1 causal collapse
\end{itemize}

These are not separate objects. They are different zoom levels of the
same RMG.

\begin{center}\rule{0.5\linewidth}{0.5pt}\end{center}

\subsubsection{3.4 RMG naturally aligns with
C\ensuremath{\Omega}MPUTER}\label{rmg-naturally-aligns-with-cux3c9mputer}

C\ensuremath{\Omega}MPUTER states:

\begin{itemize}
\tightlist
\item
  Computation = geometry
\item
  Forms = structured graphs
\item
  Shadows = observer projections
\item
  Collapse = rewrite application
\item
  Provenance = meta-structure
\item
  Universes = recursive graphs
\item
  Reasoning = tree expansion
\end{itemize}

RMG is the direct implementation of this metaphysics.

\begin{center}\rule{0.5\linewidth}{0.5pt}\end{center}

\subsubsection{3.5 RMG supports multi-universe federation
(RFC-0021)}\label{rmg-supports-multi-universe-federation-rfc-0021}

A DAG cannot link to another DAG without:

\begin{itemize}
\tightlist
\item
  collapsing them, or
\item
  creating brittle, hacky foreign pointers.
\end{itemize}

But an RMG can link:

\begin{itemize}
\tightlist
\item
  graph $\rightarrow$ graph
\item
  universe $\rightarrow$ universe
\item
  meta-graph $\rightarrow$ meta-graph
\end{itemize}

with no loss of structure or consistency.

\begin{center}\rule{0.5\linewidth}{0.5pt}\end{center}

\subsection{4. Alternatives Considered}\label{alternatives-considered}

\subsubsection{4.1 Raw DAG}\label{raw-dag}

\textbf{Rejected}: too weak, no semantic layering.

\subsubsection{4.2 DAG + JSON payloads}\label{dag-json-payloads}

\textbf{Rejected}: does not model structure or semantics; loses
composability.

\subsubsection{4.3 DAG + ``side graphs''}\label{dag-side-graphs}

\textbf{Rejected}: results in fragmentation, not a unified substrate.

\subsubsection{4.4 RMG}\label{rmg}

\textbf{Accepted}: unifies all layers; scale invariant; meta-capable;
aligns with physics; supports future universes.

\begin{center}\rule{0.5\linewidth}{0.5pt}\end{center}

\subsection{5. Consequences}\label{consequences}

Positive:

\begin{itemize}
\tightlist
\item
  Coherent substrate
\item
  Multi-layer modeling
\item
  Semantic richness
\item
  AST and graph-native transforms
\item
  Clean mapping to machine reasoning
\item
  Perfect match to C\ensuremath{\Omega}MPUTER theory
\item
  Smooth introduction of future node types
\item
  Native support for provenance
\item
  True multi-universe capabilities
\item
  Real multi-scale editing semantics
\end{itemize}

Negative:

\begin{itemize}
\tightlist
\item
  Increases conceptual complexity
\item
  Requires more formalism
\item
  Demands a stronger architecture document
\item
  Requires RMG-aware tooling
\item
  Increases burden on visualization tools
\end{itemize}

\begin{center}\rule{0.5\linewidth}{0.5pt}\end{center}

\subsection{6. Required Follow-Ups}\label{required-follow-ups}

This ADR mandates:

\begin{itemize}
\tightlist
\item
  Updating Section 3 of the Architecture Doc to describe the substrate
  as an RMG, with the DAG as the first-order projection.
\item
  Introducing RMG scale-invariance in \href{.ADR-0004.md}{ADR-0004}.
\item
  Updating collapse semantics to operate on RMG regions.
\item
  Updating provenance semantics (\href{../RFC/RFC-0016.md}{RFC-0016}) to
  attach semantic subgraphs.
\item
  Updating federation to treat universes as RMGs.
\end{itemize}

\begin{center}\rule{0.5\linewidth}{0.5pt}\end{center}

\subsection{7. Decision}\label{decision-1}

\textbf{Accepted}.

The substrate of JITOS is a Recursive Meta-Graph. The DAG is a
first-order, human-scale projection of it.

This ADR corrects the ontology of the system and sets the stage for
\href{./ADR-0004.md}{ADR-0004}.
