\section{\texorpdfstring{\textbf{ADR-0001 --- JIT as a Causal Operating
System
Kernel}}{ADR-0001 --- JIT as a Causal Operating System Kernel}}\label{adr-0001-jit-as-a-causal-operating-system-kernel}

\textbf{Status:} Proposed \textbf{Date:} 2025-11-30 \textbf{Owner:}
James Ross \textbf{Related:} JIT Whitepaper, RFC-0001--0006, CFL
(C\ensuremath{\Omega}MPUTER Fusion Layer)

\begin{center}\rule{0.5\linewidth}{0.5pt}\end{center}

\subsection{\texorpdfstring{\textbf{1.
Context}}{1. Context}}\label{context}

We have developed a deeply coherent conceptual framework and RFC set
describing:

\begin{itemize}
\tightlist
\item
  An \textbf{immutable, append-only causal DAG} as the ground truth for
  state (RFC-0001, RFC-0002).
\item
  \textbf{Shadow Working Sets (SWS)} as isolated, observer-relative
  process abstractions (RFC-0003, RFC-0019).
\item
  A \textbf{collapse operator} (commit) that turns subjective shadow
  states into objective events (RFC-0005).
\item
  A \textbf{Materialized Head (MH)} that exposes a filesystem-like
  projection for humans (RFC-0004).
\item
  A \textbf{Write-Ahead Log (WAL)} as the temporal backbone for
  deterministic replay and crash recovery (RFC-0006, RFC-0012).
\item
  An \textbf{Inversion Engine} that resolves merges, rewrites, and
  history integration without mutating past events (RFC-0005).
\item
  \textbf{RPC + ABI} giving a syscall-like interface to the Causal
  Kernel (RFC-0007, RFC-0013).
\item
  \textbf{Identity, provenance, security, sync, and federation}
  (RFC-0011, 0016, 0017, 0020, 0021).
\item
  A fusion with \textbf{C\ensuremath{\Omega}MPUTER's metaphysical model} where computation
  is geometry and causality (CFL, RFC-0024).
\end{itemize}

Up to now, this has been presented as:

\begin{itemize}
\tightlist
\item
  ``A Git inversion layer''
\item
  ``A causal database that speaks Git''
\item
  ``A provenance engine''
\item
  ``A post-file compute substrate''
\end{itemize}

But all of these are underselling what the design actually describes.

The architecture that has emerged isn't ``a better VCS.''

It is structurally, functionally, and philosophically equivalent to an
\textbf{operating system kernel}, with:

\begin{itemize}
\tightlist
\item
  process isolation $\rightarrow$ SWS
\item
  memory model $\rightarrow$ causal DAG + SWS memory model
\item
  filesystem $\rightarrow$ MH projection
\item
  syscalls $\rightarrow$ JIT RPC + ABI
\item
  scheduler \& consistency $\rightarrow$ Inversion Engine + Message Plane
\item
  logging \& time $\rightarrow$ WAL
\item
  identity \& permissions $\rightarrow$ AIS + Security Model
\item
  boot \& recovery $\rightarrow$ JITOS boot RFC
\item
  multi-node \& federation $\rightarrow$ sync + MUFP
\end{itemize}

This ADR decides:

\begin{quote}
\textbf{We will explicitly treat JIT not as a library or protocol, but
as a full-blown OS kernel --- JITOS.}
\end{quote}

\begin{center}\rule{0.5\linewidth}{0.5pt}\end{center}

\subsection{\texorpdfstring{\textbf{2.
Decision}}{2. Decision}}\label{decision}

We formally decide:

\begin{quote}
\textbf{JIT (Git Inversion Tech) is the kernel of a new causal operating
system: JITOS.}
\end{quote}

\subsubsection{\texorpdfstring{\textbf{This
implies:}}{This implies:}}\label{this-implies}

\begin{enumerate}
\def\labelenumi{\arabic{enumi}.}
\tightlist
\item
  \textbf{JITD is the Kernel Process (jitd / jitosd)}

  \begin{itemize}
  \tightlist
  \item
    It is long-running, privileged, and authoritative over:

    \begin{itemize}
    \tightlist
    \item
      The DAG (truth)
    \item
      SWS lifecycle
    \item
      Collapse/commit
    \item
      MH consistency
    \item
      WAL replay
    \item
      Ref management
    \item
      Sync and federation
    \item
      Security enforcement
    \end{itemize}
  \end{itemize}
\item
  \textbf{SWS are Processes}

  \begin{itemize}
  \tightlist
  \item
    Every meaningful ``unit of work'' (human or machine) executes inside
    a Shadow Working Set.
  \item
    SWS becomes the primary process abstraction of JITOS.
  \end{itemize}
\item
  \textbf{The DAG is Unified Memory + History}

  \begin{itemize}
  \tightlist
  \item
    The causal DAG is not ``just a log.''
  \item
    It is the \textbf{canonical memory model} of JITOS:

    \begin{itemize}
    \tightlist
    \item
      immutable state
    \item
      perfect replay
    \item
      cross-cutting history
    \item
      global source of truth
    \end{itemize}
  \end{itemize}
\item
  \textbf{Materialized Head is the Filesystem Projection}

  \begin{itemize}
  \tightlist
  \item
    The filesystem is \emph{not} the state.
  \item
    It is a derived projection of the DAG for human tooling and
    compatibility.
  \end{itemize}
\item
  \textbf{JIT RPC + ABI is the Syscall Surface}

  \begin{itemize}
  \tightlist
  \item
    All external tools (CLIs, agents, IDEs, services) interact with
    JITOS via:

    \begin{itemize}
    \tightlist
    \item
      structured RPC
    \item
      stable binary ABI
    \item
      versioned capabilities
    \end{itemize}
  \end{itemize}
\item
  \textbf{JITOS Is The Primary Runtime Environment}

  \begin{itemize}
  \tightlist
  \item
    This is not ``just infra'' under another OS layer.
  \item
    JITOS is meant to be:

    \begin{itemize}
    \tightlist
    \item
      hostable on conventional OSes (Linux/macOS/Windows) initially
    \item
      but architected as a \textbf{kernel in its own right} for future
      more-native deployment.
    \end{itemize}
  \end{itemize}
\end{enumerate}

\begin{center}\rule{0.5\linewidth}{0.5pt}\end{center}

\subsection{\texorpdfstring{\textbf{3.
Rationale}}{3. Rationale}}\label{rationale}

\subsubsection{\texorpdfstring{\textbf{3.1 Conceptual
Integrity}}{3.1 Conceptual Integrity}}\label{conceptual-integrity}

The system we designed has all the properties of an OS kernel:

\begin{itemize}
\tightlist
\item
  It provides isolation (SWS).
\item
  It mediates state changes (Inversion Engine).
\item
  It defines a memory model (DAG + SWS-MM).
\item
  It defines execution semantics (collapse, Message Plane).
\item
  It mediates I/O and views (MH, RPC).
\item
  It boots, replays, and recovers (WAL, boot RFC).
\item
  It enforces security and identity (AIS, security RFC).
\item
  It syncs distributed state (Sync, MUFP).
\end{itemize}

Calling it ``a service'' or ``a library'' underdescribes it and weakens
the design.

Naming it what it truly is --- a kernel --- clarifies:

\begin{itemize}
\tightlist
\item
  how subsystems relate
\item
  what guarantees must be provided
\item
  how tools should integrate
\item
  how future extensions should be framed
\end{itemize}

\subsubsection{\texorpdfstring{\textbf{3.2 Evolution \&
Adoption}}{3.2 Evolution \& Adoption}}\label{evolution-adoption}

By framing JIT as:

\begin{quote}
\textbf{``The kernel of a causal OS that also speaks Git''}
\end{quote}

\ldots we get:

\begin{itemize}
\tightlist
\item
  A path to adopt it \textbf{incrementally}:

  \begin{itemize}
  \tightlist
  \item
    First as a Git backend
  \item
    Then as a provenance/logging substrate
  \item
    Then as an execution \& agent orchestration layer
  \item
    Then as the foundation for new apps/languages
  \end{itemize}
\item
  A clear mental model for ecosystem builders:

  \begin{itemize}
  \tightlist
  \item
    ``This is my OS for agent-native, causal, multi-tenant compute.''
  \end{itemize}
\item
  A better alignment with C\ensuremath{\Omega}MPUTER's cosmology:

  \begin{itemize}
  \tightlist
  \item
    C\ensuremath{\Omega}MPUTER = theory of computation as causal geometry
  \item
    JITOS = practical instantiation of that theory
  \end{itemize}
\end{itemize}

\subsubsection{\texorpdfstring{\textbf{3.3 Strategic
Positioning}}{3.3 Strategic Positioning}}\label{strategic-positioning}

This decision:

\begin{itemize}
\tightlist
\item
  differentiates JITOS from:

  \begin{itemize}
  \tightlist
  \item
    databases
  \item
    message buses
  \item
    VCS-only tools
  \item
    simple logs
  \end{itemize}
\item
  positions it as:

  \begin{itemize}
  \tightlist
  \item
    the \textbf{substrate} for post-file, agent-native computing
  \item
    a ``Linux for the causal age''
  \end{itemize}
\end{itemize}

\begin{center}\rule{0.5\linewidth}{0.5pt}\end{center}

\subsection{\texorpdfstring{\textbf{4. Alternatives
Considered}}{4. Alternatives Considered}}\label{alternatives-considered}

\subsubsection{\texorpdfstring{\textbf{4.1 ``JIT as a Git Backend
Only''}}{4.1 ``JIT as a Git Backend Only''}}\label{jit-as-a-git-backend-only}

\begin{itemize}
\tightlist
\item
  Pros:

  \begin{itemize}
  \tightlist
  \item
    Easier story
  \item
    Less intimidating
  \end{itemize}
\item
  Cons:

  \begin{itemize}
  \tightlist
  \item
    Severely underrepresents the capabilities
  \item
    Confuses architecture (where ``kernel-like'' features come from)
  \item
    Undermines the OS-level abstractions like SWS, WAL, MH
  \end{itemize}
\end{itemize}

Rejected because it misframes the system.

\begin{center}\rule{0.5\linewidth}{0.5pt}\end{center}

\subsubsection{\texorpdfstring{\textbf{4.2 ``JIT as a Database +
Framework''}}{4.2 ``JIT as a Database + Framework''}}\label{jit-as-a-database-framework}

\begin{itemize}
\tightlist
\item
  JIT as ``a causal DB with a nice API for agents + Git support''
\item
  Pros:

  \begin{itemize}
  \tightlist
  \item
    Comfortable mental model for many developers
  \end{itemize}
\item
  Cons:

  \begin{itemize}
  \tightlist
  \item
    Fails to emphasize:

    \begin{itemize}
    \tightlist
    \item
      process model
    \item
      memory model
    \item
      boot \& recovery semantics
    \item
      security as a systemic property
    \end{itemize}
  \item
    Leads to misuse as ``just another DB'' instead of \textbf{defining
    the runtime}.
  \end{itemize}
\end{itemize}

Rejected because it hides its true role.

\begin{center}\rule{0.5\linewidth}{0.5pt}\end{center}

\subsubsection{\texorpdfstring{\textbf{4.3 ``JIT as a Pure Library /
SDK''}}{4.3 ``JIT as a Pure Library / SDK''}}\label{jit-as-a-pure-library-sdk}

\begin{itemize}
\tightlist
\item
  Provides types, clients, protocols
\item
  Leaves everything else up to the host app
\end{itemize}

Rejected because:

\begin{itemize}
\tightlist
\item
  We need a \textbf{single authoritative kernel} to enforce causal
  invariants.
\item
  Library approaches cannot reliably enforce:

  \begin{itemize}
  \tightlist
  \item
    WAL ordering
  \item
    global DAG integrity
  \item
    multi-agent isolation
  \item
    collapse semantics
  \end{itemize}
\end{itemize}

\begin{center}\rule{0.5\linewidth}{0.5pt}\end{center}

\subsection{\texorpdfstring{\textbf{5.
Consequences}}{5. Consequences}}\label{consequences}

\subsubsection{\texorpdfstring{\textbf{5.1
Positive}}{5.1 Positive}}\label{positive}

\begin{itemize}
\tightlist
\item
  \textbf{Clarity:} Everyone understands JIT as a kernel.
\item
  \textbf{Coherence:} All subsystems align with OS semantics.
\item
  \textbf{Extensibility:} Future features (scheduler, agent economy,
  etc.) fit naturally.
\item
  \textbf{Research Value:} JITOS becomes an explicit research target
  (Causal OS).
\item
  \textbf{Developer Understanding:} It becomes easier to teach and
  document.
\end{itemize}

\subsubsection{\texorpdfstring{\textbf{5.2 Negative /
Tradeoffs}}{5.2 Negative / Tradeoffs}}\label{negative-tradeoffs}

\begin{itemize}
\tightlist
\item
  \textbf{Increased Ambition:} This is harder than ``a Git backend.''
\item
  \textbf{Expectations:} Calling it an OS kernel implies a high bar of
  robustness and rigor.
\item
  \textbf{Adoption Path:} Some will be intimidated by the ``OS''
  framing.
\item
  \textbf{Formal Verification Pressure:} The more ``foundational'' it
  is, the more pressure for proofs.
\end{itemize}

\subsubsection{\texorpdfstring{\textbf{5.3 Required
Follow-Ups}}{5.3 Required Follow-Ups}}\label{required-follow-ups}

This ADR implies:

\begin{itemize}
\tightlist
\item
  The Architecture Doc \textbf{MUST} be structured \textbf{as a kernel
  design doc}:

  \begin{itemize}
  \tightlist
  \item
    Processes (SWS)
  \item
    Memory model (DAG + SWS-MM)
  \item
    I/O model (MH, RPC)
  \item
    Execution model (collapse + inversion)
  \item
    Scheduling/coordination (Message Plane, future scheduler)
  \item
    Storage model (tiering, isolation)
  \item
    Security model (AIS, permissions)
  \item
    Boot sequence
  \item
    Federation
  \end{itemize}
\item
  RFCs should be grouped under these kernel subsystems in the Arch Doc.
\end{itemize}

\begin{center}\rule{0.5\linewidth}{0.5pt}\end{center}

\subsection{\texorpdfstring{\textbf{6.
Decision}}{6. Decision}}\label{decision-1}

\textbf{Accepted.}

From this point forward:

\begin{itemize}
\tightlist
\item
  \textbf{JIT is referred to as the kernel of JITOS}, the causal
  operating system.
\item
  All further ADRs and Architecture Doc sections are written under that
  framing.
\item
  The JITOS Architecture Document will treat JITD as the kernel, not as
  a ``service'' or ``tool.''
\end{itemize}

\begin{center}\rule{0.5\linewidth}{0.5pt}\end{center}

\section{\texorpdfstring{\textbf{C\ensuremath{\Omega}MPUTER {\textbullet}
JITOS}}{C\ensuremath{\Omega}MPUTER {\textbullet} JITOS}}\label{cux3c9mputer-jitos}

{\textcopyright} 2025 James Ross {\textbullet} \href{https://flyingrobots.dev}{Flying {\textbullet} Robots} All
Rights Reserved
