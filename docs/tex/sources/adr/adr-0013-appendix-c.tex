\section{\texorpdfstring{\textbf{Appendix C --- JS-ABI v1.0 Deterministic Compliance Gauntlet}}{Appendix C --- JS-ABI v1.0 Deterministic Compliance Gauntlet}}\label{adr-0013-appendix-c}

This appendix defines a \textbf{minimum compliance suite} (the
``Gauntlet'') for implementations of JS-ABI v1.0.

An implementation \textbf{MUST} pass all tests in this appendix to be
considered JS-ABI v1.0 compliant for the purposes of WAL, hashing, and
deterministic replay.

The Gauntlet is divided into:

\begin{itemize}
\tightlist
  \item Encoder tests (behavior of the CBOR encoder),
  \item Decoder tests (behavior of the CBOR decoder),
  \item End-to-end tests (behavior of the kernel logical clock and WAL).
\end{itemize}

Test vector IDs (TV1, TV2, etc.) refer to Appendix~\ref{sec:appendix-test-vectors}.

\subsection{\texorpdfstring{\textbf{C.1 Encoder Compliance Tests}}{C.1 Encoder Compliance Tests}}\label{encoder-compliance-tests}

\subsubsection*{EC-01: Exact Encoding of Handshake (TV1)}

\textbf{Input (logical form):} the JSON structure for the handshake in
Appendix B, Section B.1.

\textbf{Requirement:}

\begin{itemize}
\tightlist
  \item When serialized by the implementation's CBOR encoder under
        JS-ABI v1.0's deterministic profile, the payload bytes
        \textbf{MUST EXACTLY MATCH} the hex string for TV1.
  \item The length field in the packet header \textbf{MUST} be
        \texttt{0x00000071} (113 decimal).
\end{itemize}

If the encoder produces any deviating CBOR bytes (even if semantically
equivalent), it \textbf{fails} EC-01.

\subsubsection*{EC-02: Exact Encoding of Error (TV3)}

\textbf{Input (logical form):} the JSON structure for the error in
Appendix B, Section B.2.

\textbf{Requirement:}

\begin{itemize}
\tightlist
  \item When serialized by the encoder, the payload
        \textbf{MUST EXACTLY MATCH} the hex string for TV3.
  \item The length field in the packet header \textbf{MUST} be
        \texttt{0x00000076} (118 decimal).
\end{itemize}

Any deviation in encoding (different integer widths, map ordering,
string lengths, etc.) \textbf{fails} EC-02.

\subsubsection*{EC-03: Integer Minimal Width}

\textbf{Input:} the following logical values encoded as CBOR integers:

\begin{itemize}
\tightlist
  \item 0, 1, 10, 23
  \item 24, 100, 255
  \item 256, 1000, 65535
  \item 65536, \(2^{32}-1\) (4294967295)
\end{itemize}

\textbf{Requirement:}

\begin{itemize}
\tightlist
  \item For each value, the encoder \textbf{MUST} emit the shortest
        possible CBOR integer form per RFC~8949 Preferred Serialization
        (major type 0 / 1 with minimal additional length).
  \item No value may be encoded with a larger-than-necessary integer
        width (e.g., 23 may not be encoded using a 2-byte or 4-byte
        integer).
\end{itemize}

Any larger-than-minimal integer representation \textbf{fails} EC-03.

\subsubsection*{EC-04: Integer vs Float Encoding}

\textbf{Input:} logical numeric values:
0, 1, -1, 42, 0.0, 1.0, -1.0, 0.5, 1.5.

\textbf{Requirement:}

\begin{itemize}
\tightlist
  \item 0, 1, -1, 42, 0.0, 1.0, -1.0 \textbf{MUST} be encoded as
        integers (major type 0 or 1), not as floating-point.
  \item 0.5 and 1.5 \textbf{MUST} be encoded as floating-point using the
        smallest width that preserves the value exactly (typically half
        or single precision depending on implementation).
  \item No numeric value that is mathematically an integer and within
        CBOR integer range may be encoded as floating-point.
\end{itemize}

If the encoder emits any float where an integer is required, or uses a
wider float than necessary, it \textbf{fails} EC-04.

\subsubsection*{EC-05: Canonical Map Ordering}

\textbf{Input:} CBOR maps with keys chosen to exercise canonical
ordering, e.g.:

\begin{verbatim}
{
  "a": 1,
  "b": 2,
  "aa": 3,
  "Z": 4,
  "op": "test",
  "ts": 0,
  "payload": {}
}
\end{verbatim}

\textbf{Requirement:} keys must be ordered by their full CBOR encoding
(bytewise increasing), matching the ordering implied by the canonical
encodings in TV1 and TV3. Any mismatch \textbf{fails} EC-05.

\subsubsection*{EC-06: No Tags, No Indefinite Length}

\textbf{Input:} any \texttt{OpEnvelope} payload.

\textbf{Requirement:} the encoder must not emit CBOR tags (major type
6), indefinite-length strings, arrays, or maps, nor the break code
(\texttt{0xff}). Any such encoding \textbf{fails} EC-06.

\subsection{\texorpdfstring{\textbf{C.2 Decoder Compliance Tests}}{C.2 Decoder Compliance Tests}}\label{decoder-compliance-tests}

\subsubsection*{DC-01: Accept Canonical TV1 and TV3}

\textbf{Input:} the exact CBOR payload bytes of TV1 and TV3.

\textbf{Requirement:} the decoder must successfully decode both payloads
to the expected logical structures. Failure to decode these encodings
fails DC-01.

\subsubsection*{DC-02: Reject Indefinite-Length Encodings}

\textbf{Input:} variants of TV1 in which maps, arrays, or strings use
indefinite length with a break code but identical logical content.

\textbf{Requirement:} the decoder must treat any use of
indefinite-length encodings as non-compliant and reject the packet as
JS-ABI v1.0 payload. Accepting such a payload \textbf{fails} DC-02.

\subsubsection*{DC-03: Reject Non-Canonical Integers}

\textbf{Input:} variants of TV1 in which an integer (e.g.
\texttt{ts = 0} or \texttt{client\_version = 1}) is encoded using a
wider-than-necessary integer format.

\textbf{Requirement:} the decoder may parse such payloads for
diagnostics, but must treat them as non-compliant with JS-ABI v1.0 and
must not accept them into WAL or deterministic replay. Treating such a
payload as normal input \textbf{fails} DC-03.

\subsubsection*{DC-04: Reject Tags}

\textbf{Input:} variants of TV3 that introduce a CBOR tag around one of
the fields while preserving logical content.

\textbf{Requirement:} any CBOR tag in the payload must cause the decoder
to reject the packet as JS-ABI v1.0 input. Accepting a tagged payload
\textbf{fails} DC-04.

\subsubsection*{DC-05: Reject Duplicate Map Keys}

\textbf{Input:} variants of TV1 or TV3 in which the top-level map or the
\texttt{payload} map contain duplicate keys.

\textbf{Requirement:} duplicate keys must be treated as a violation of
the deterministic encoding profile and cause the payload to be
rejected. Silently choosing one entry and continuing \textbf{fails}
DC-05.

\subsubsection*{DC-06: Map Key Ordering Independence (Strict Mode)}

\textbf{Input:} variants of TV1 where map keys are emitted in
non-canonical order while preserving logical fields.

\textbf{Requirement:} in strict JS-ABI v1.0 mode, such non-canonical
payloads must be rejected for use in WAL and deterministic replay. Using
them as if they were canonical input \textbf{fails} DC-06. (A lenient
tooling mode is outside JS-ABI v1.0 compliance.)

\subsection{\texorpdfstring{\textbf{C.3 End-to-End Logical Clock and WAL Tests}}{C.3 End-to-End Logical Clock and WAL Tests}}\label{end-to-end-tests}

\subsubsection*{EE-01: Server-Assigned Timestamps}

\textbf{Scenario:}

\begin{enumerate}
\tightlist
  \item Start with an empty kernel and WAL.
  \item A client sends two requests (e.g., a handshake and a
        state-mutating operation) with \texttt{ts = 0}.
  \item The server processes both and appends the resulting operations
        to the WAL.
\end{enumerate}

\textbf{Requirement:} WAL entries must have strictly increasing
\texttt{ts} values, assigned by the kernel logical clock (not copied
from the client). Any non-monotonic or zero \texttt{ts} in the WAL
\textbf{fails} EE-01.

\subsubsection*{EE-02: Monotonicity Under Concurrency}

\textbf{Scenario:}

\begin{enumerate}
\tightlist
  \item Multiple clients concurrently issue state-mutating operations
        with arbitrary \texttt{ts} values (including 0 and stale
        values).
  \item The server interleaves processing and appends all committed
        operations to the WAL.
\end{enumerate}

\textbf{Requirement:} WAL entries must have strictly increasing
\texttt{ts} values forming a total order consistent with execution. Any
duplicate or non-monotonic \texttt{ts} values \textbf{fail} EE-02.

\subsubsection*{EE-03: Deterministic Replay}

\textbf{Scenario:}

\begin{enumerate}
\tightlist
  \item From a known initial state, execute a sequence of valid JS-ABI
        operations, recording the resulting WAL and final state.
  \item Reinitialize the kernel to the same initial state.
  \item Replay the WAL entries in strictly increasing \texttt{ts} order,
        using the recorded payloads verbatim.
\end{enumerate}

\textbf{Requirement:} the resulting kernel state and any observable
\texttt{OpEnvelope} sequence must match the original. Failure to
reproduce the same state or sequence \textbf{fails} EE-03.

\subsection{\texorpdfstring{\textbf{C.4 Compliance Definition}}{C.4 Compliance Definition}}\label{compliance-definition}

An implementation is JS-ABI v1.0 \emph{deterministic-compliant} for WAL
and replay iff:

\begin{itemize}
\tightlist
  \item It passes all encoder tests EC-01 through EC-06,
  \item It passes all decoder tests DC-01 through DC-06, and
  \item It passes all end-to-end tests EE-01 through EE-03.
\end{itemize}

Failing any test means the implementation may still interoperate in a
best-effort fashion but cannot be trusted for cross-language
deterministic WAL, hashing, or replay, and must not be advertised as
JS-ABI v1.0 compliant.
