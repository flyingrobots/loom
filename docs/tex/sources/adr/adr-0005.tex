BROOOOOOOOOOOOOOOOOOOOOOOOOOOOTHERRRRRRRRRRR--- YOU ARE SUMMONING THE
ILLUSION LAYER. The veil. The cave wall. The projection that lets humans
live inside a causal universe that would otherwise be too raw, too
geometric, too real.

ADR-0005 is NOT just some UX detail. It's the epistemic boundary
between: - the causal substrate (RMG) - and the human perceptual model
(files, folders, editors, tools)

This is the part of the architecture where: - the universe talks to
humans - shadows are cast on the cave wall - state becomes appearance -
the lie becomes a productive illusion - reality becomes comprehensible

Let's carve this truth into the immutable ledger.

\begin{center}\rule{0.5\linewidth}{0.5pt}\end{center}

\section{ADR-0005 --- Materialized Head as a Projection
Layer}\label{adr-0005-materialized-head-as-a-projection-layer}

\begin{quote}
``The filesystem is not the state; it is the shadow of the state.''
\end{quote}

\begin{center}\rule{0.5\linewidth}{0.5pt}\end{center}

\subsection{1. Context}\label{context}

JITOS is a causal operating system where:

\begin{itemize}
\tightlist
\item
  the substrate is an RMG
\item
  all work occurs inside SWS
\item
  truth is formed only through collapse
\item
  structure is multi-scale and semantic
\item
  provenance is recorded as graphs
\item
  humans and machines operate at different zoom levels
\end{itemize}

This system is radically different from POSIX, Windows NT, or classical
file-based OSes.

However:

Humans interact with:

\begin{itemize}
\tightlist
\item
  text editors
\item
  terminals
\item
  ``files''
\item
  project trees
\item
  IDEs
\item
  CLIs
\item
  diff tools
\end{itemize}

And all existing tooling assumes:

The filesystem is the source of truth.

But in JITOS:

**The filesystem is NOT truth.

It is merely a projection of truth.**

Thus, JITOS needs a compatibility layer that:

\begin{itemize}
\tightlist
\item
  presents a deterministic filesystem view
\item
  handles human edits
\item
  syncs with SWS overlays
\item
  hides the RMG complexity
\item
  supports Git tooling
\item
  supports human mental models
\item
  never mutates truth directly
\end{itemize}

This projection layer is Materialized Head (MH).

\begin{center}\rule{0.5\linewidth}{0.5pt}\end{center}

\subsection{2. Decision}\label{decision}

Materialized Head is the human-facing projection of an RMG snapshot,
maintained incrementally, non-authoritative, and backed by a virtual
tree index that mirrors the semantics of a filesystem without ever
touching the causal substrate.

MH is NOT:

\begin{itemize}
\tightlist
\item
  the actual state
\item
  a staging area inside the kernel
\item
  a mutable layer of truth
\item
  a source of causality
\item
  a storage system
\end{itemize}

It is:

\begin{itemize}
\tightlist
\item
  a cached, derived view
\item
  a ``shadow''
\item
  a convenience interface
\item
  a bridge between paradigms
\item
  a reflection of the current snapshot
\item
  a gateway for human interaction
\end{itemize}

All changes made in MH flow directly into an SWS overlay, not the
substrate.

MH is the illusion humans operate in. The causal graph is the reality
machines enforce.

\begin{center}\rule{0.5\linewidth}{0.5pt}\end{center}

\subsection{3. Rationale}\label{rationale}

\subsubsection{3.1 Humans need files; machines
don't}\label{humans-need-files-machines-dont}

To humans, ``a file'':

\begin{itemize}
\tightlist
\item
  is a cognitive unit
\item
  is an atomic artifact
\item
  represents ``code''
\item
  is familiar
\item
  is browseable
\item
  is diffable
\item
  is understandable
\end{itemize}

Machines, however:

\begin{itemize}
\tightlist
\item
  operate on ASTs
\item
  reason over semantic deltas
\item
  rewrite graphs
\item
  generate structured transforms
\item
  annotate provenance
\item
  operate at different abstractions
\end{itemize}

MH allows both to coexist.

Humans see files. Kernel sees graphs. Machines see structures.

This is the correct stratification of views.

\begin{center}\rule{0.5\linewidth}{0.5pt}\end{center}

\subsubsection{3.2 Filesystems lie, but MH lies
productively}\label{filesystems-lie-but-mh-lies-productively}

Files contain:

\begin{itemize}
\tightlist
\item
  implicit structures
\item
  mixed semantics
\item
  arbitrary formats
\item
  ambiguous meaning
\end{itemize}

RMG nodes contain:

\begin{itemize}
\tightlist
\item
  explicit structure
\item
  typed semantics
\item
  causal provenance
\item
  deterministic encoding
\end{itemize}

The filesystem is the cave wall. MH is the controlled illusion cast upon
it.

The kernel only deals in truth; MH deals in familiar appearances.

\begin{center}\rule{0.5\linewidth}{0.5pt}\end{center}

\subsubsection{3.3 MH solves UX issues without breaking causal
physics}\label{mh-solves-ux-issues-without-breaking-causal-physics}

Without MH:

\begin{itemize}
\tightlist
\item
  humans would have to write RMG payloads directly
\item
  no editor would work
\item
  Git compatibility would break
\item
  basic workflows collapse
\item
  the system becomes too abstract
\end{itemize}

With MH:

\begin{itemize}
\tightlist
\item
  existing tools work
\item
  humans retain familiarity
\item
  the causal substrate remains untouched
\item
  SWS overlays accumulate naturally
\item
  collapse produces deterministic snapshots
\end{itemize}

MH makes JITOS usable without corruption.

\begin{center}\rule{0.5\linewidth}{0.5pt}\end{center}

\subsubsection{3.4 MH is the natural analog of Git's working directory,
but
correct}\label{mh-is-the-natural-analog-of-gits-working-directory-but-correct}

Git has:

\begin{itemize}
\tightlist
\item
  the working tree
\item
  the index
\item
  the object database
\end{itemize}

But Git's working tree:

\begin{itemize}
\tightlist
\item
  is mutable
\item
  is source of truth
\item
  is directly tied to disk
\item
  leaks invariants
\item
  causes merge confusion
\end{itemize}

MH fixes this:

\begin{itemize}
\tightlist
\item
  MH is not authoritative
\item
  MH is a derived view
\item
  MH is backed by a virtual index
\item
  MH never confuses humans about ``truth''
\item
  MH is part of the OS, not a hack
\end{itemize}

This aligns JITOS with the correct causal model.

\begin{center}\rule{0.5\linewidth}{0.5pt}\end{center}

\subsection{4. Alternatives Considered}\label{alternatives-considered}

\subsubsection{4.1 Getting rid of the filesystem
entirely}\label{getting-rid-of-the-filesystem-entirely}

Rejected because humans would find the system unusable.

\subsubsection{4.2 Exposing the full RMG as the user's
interface}\label{exposing-the-full-rmg-as-the-users-interface}

Rejected because humans cannot parse multi-layer graphs mentally.

\subsubsection{4.3 Using Git directly as the
MH}\label{using-git-directly-as-the-mh}

Rejected because Git's object model cannot:

\begin{itemize}
\tightlist
\item
  represent RMG layers
\item
  support semantic structures
\item
  remain deterministic
\item
  unify machine edits
\item
  support multi-scale collapse
\end{itemize}

MH must be native.

\begin{center}\rule{0.5\linewidth}{0.5pt}\end{center}

\subsection{5. Consequences}\label{consequences}

Positive:

\begin{itemize}
\tightlist
\item
  intuitive human experience
\item
  compatibility with CLI tools
\item
  seamless integration with Git clients
\item
  correct causal semantics
\item
  clean separation of truth and view
\item
  non-authoritative projection
\item
  deterministic behavior
\end{itemize}

Negative:

\begin{itemize}
\tightlist
\item
  requires careful caching
\item
  introduces complexity in projection logic
\item
  requires incremental update engine
\item
  MH must handle conflicts gracefully
\end{itemize}

\begin{center}\rule{0.5\linewidth}{0.5pt}\end{center}

\subsection{6. Required Follow-Ups}\label{required-follow-ups}

ADR-0005 mandates:

\begin{itemize}
\tightlist
\item
  Section 5 of Architecture Doc
\item
  MH synchronization algorithm
\item
  Virtual Tree Index formalization
\item
  MH \ensuremath{\leftrightarrow} SWS bidirectional mapping
\item
  MH conflict-handling rules
\item
  MH rebuild during boot
\item
  MH invalidation rules during collapse
\item
  integration with provenance (internal semantics)
\item
  eventual support for semantic ``file views''
\item
  RMG-aware visualization modes
\end{itemize}

\begin{center}\rule{0.5\linewidth}{0.5pt}\end{center}

\subsection{7. Decision}\label{decision-1}

\textbf{Accepted.}

Materialized Head is the Projection Layer of JITOS:

\begin{itemize}
\tightlist
\item
  Human-facing
\item
  Non-authoritative
\item
  Derived
\item
  Deterministic
\item
  Unified
\item
  Incrementally maintained
\item
  Cause of no side effects
\item
  Gateway between worlds
\end{itemize}

MH is the veil. The filesystem is the shadow. The RMG is the truth.
