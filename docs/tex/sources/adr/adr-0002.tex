\section{ADR-0002 --- The Shadow Working Set (SWS) as the Process
Model}\label{adr-0002-the-shadow-working-set-sws-as-the-process-model}

We're about to define the core abstraction around which the entire
causal OS orbits.

This is the equivalent of defining ``processes'' in Unix, ``threads'' in
Linux, ``actors'' in Erlang, ``isolates'' in Dart, or ``greenlets'' in
Go\ldots{}

Except ours are:

\begin{itemize}
\tightlist
\item
  metaphysical
\item
  causal
\item
  semantic
\item
  multi-agent
\item
  observer-relative
\item
  and aligned with the physics of C\ensuremath{\Omega}MPUTER.
\end{itemize}

This ADR is absolutely foundational. This is the one that future
implementers, engineers, researchers, philosophers, and AI-agent
designers will reference.

Let's carve it into the architecture.

\begin{center}\rule{0.5\linewidth}{0.5pt}\end{center}

\subsection{1. Context}\label{context}

Operating systems require:

\begin{itemize}
\tightlist
\item
  a process model
\item
  a unit of execution
\item
  a boundary for isolation
\item
  a mechanism for concurrency
\item
  a container for state
\item
  a sandbox for computation
\item
  a context for identity and permissions
\end{itemize}

Traditional systems use:

\begin{itemize}
\tightlist
\item
  processes
\item
  threads
\item
  green threads
\item
  actors
\item
  fibers
\item
  isolates
\end{itemize}

But all of these assume:

\begin{itemize}
\tightlist
\item
  mutable state
\item
  shared memory
\item
  global clocks
\item
  mutable files
\item
  linear time
\item
  immediate access to global truth
\end{itemize}

These assumptions break under:

\begin{itemize}
\tightlist
\item
  multi-agent systems
\item
  distributed computation
\item
  LLM-based reasoning
\item
  speculative editing
\item
  scientific reproducibility
\item
  concurrency under relativistic constraints
\item
  causal consistency
\item
  immutable past semantics
\end{itemize}

JITOS needs a process model that:

\begin{itemize}
\tightlist
\item
  fits causal invariants
\item
  supports observer-relative computation
\item
  isolates concurrent edits
\item
  supports multi-agent workflows
\item
  handles speculative state
\item
  collapses deterministically
\item
  leaves history untouched
\item
  is ephemeral yet formally defined
\item
  maps directly to the C\ensuremath{\Omega}MPUTER ontology
\end{itemize}

This leads naturally to the concept we've been dancing around:

Shadow Working Sets (SWS)

as the process abstraction of the causal OS.

This ADR formalizes SWS as the canonical, singular process model of
JITOS.

\begin{center}\rule{0.5\linewidth}{0.5pt}\end{center}

\subsection{2. Decision}\label{decision}

\begin{quote}
In JITOS, the process abstraction is the Shadow Working Set (SWS).

There are no threads, no processes, no coroutines, no fibers.

All computation---human or machine---occurs inside an SWS: a temporary,
isolated, observer-relative branch of the causal universe that collapses
into a new immutable event.
\end{quote}

This includes:

\begin{itemize}
\tightlist
\item
  human edits
\item
  code modifications
\item
  LLM reasoning
\item
  CI task execution
\item
  refactoring
\item
  analyses
\item
  semantic transformations
\item
  simulation step evaluations
\item
  distributed agent interaction
\end{itemize}

Everything that ``runs'' runs in a Shadow.

SWS is the JITOS notion of a process.

\begin{center}\rule{0.5\linewidth}{0.5pt}\end{center}

\subsection{3. Rationale}\label{rationale}

\subsubsection{3.1 SWS naturally solves
concurrency}\label{sws-naturally-solves-concurrency}

Shadows are isolated by definition:

\begin{itemize}
\tightlist
\item
  no shared mutable state
\item
  no conflicts until collapse
\item
  no global lock contention
\item
  no file-level races
\item
  no concurrency hazards
\item
  each agent sees a subjective local world
\end{itemize}

This mirrors special relativity rather than the POSIX memory model.

\begin{center}\rule{0.5\linewidth}{0.5pt}\end{center}

\subsubsection{3.2 SWS matches the metaphysics of
C\ensuremath{\Omega}MPUTER}\label{sws-matches-the-metaphysics-of-cux3c9mputer}

C\ensuremath{\Omega}MPUTER states:

\begin{itemize}
\tightlist
\item
  computation occurs in observer-specific frames
\item
  collapse events define reality
\item
  shadows represent potential worlds
\item
  truth is an immutable DAG
\end{itemize}

SWS are the practical implementation of this metaphysics.

\begin{center}\rule{0.5\linewidth}{0.5pt}\end{center}

\subsubsection{3.3 SWS integrate perfectly with the
DAG}\label{sws-integrate-perfectly-with-the-dag}

\begin{itemize}
\tightlist
\item
  Base snapshot = initial state
\item
  Overlays = speculative deltas
\item
  Merge/Collapse = DAG event
\item
  Destroy = no DAG mutation
\item
  Provenance = optional semantic overlay
\end{itemize}

Everything aligns with invariant-based DAG logic.

\begin{center}\rule{0.5\linewidth}{0.5pt}\end{center}

\subsubsection{3.4 SWS support multi-agent systems by
design}\label{sws-support-multi-agent-systems-by-design}

Each agent gets:

\begin{itemize}
\tightlist
\item
  its own isolated shadow
\item
  no interference
\item
  no need for locks
\item
  no global coordination
\item
  purely local computation
\item
  deterministic merge semantics
\end{itemize}

Perfect for:

\begin{itemize}
\tightlist
\item
  LLMs
\item
  autonomous agents
\item
  CI pipelines
\item
  semantic bots
\item
  analysis tools
\item
  GUI editors
\item
  batch jobs
\item
  long-range simulations
\end{itemize}

\begin{center}\rule{0.5\linewidth}{0.5pt}\end{center}

\subsubsection{3.5 SWS unify humans and
machines}\label{sws-unify-humans-and-machines}

Humans edit files via MH $\rightarrow$ SWS. Machines edit nodes via RPC $\rightarrow$ SWS. Both
are equivalent actors.

This makes JITOS the first OS where:

humans and AI share the same process abstraction.

\begin{center}\rule{0.5\linewidth}{0.5pt}\end{center}

\subsection{4. Alternatives Considered}\label{alternatives-considered}

\subsubsection{4.1 Traditional processes +
threads}\label{traditional-processes-threads}

Rejected because:

\begin{itemize}
\tightlist
\item
  imply shared memory
\item
  violate causal semantics
\item
  not reproducible
\item
  nondeterministic
\item
  difficult to reason about
\item
  incompatible with DAG-based truth
\end{itemize}

\subsubsection{4.2 CRDT actors}\label{crdt-actors}

Rejected because:

\begin{itemize}
\tightlist
\item
  CRDTs assume distributed mutable state
\item
  we do not mutate state at all
\item
  convergence semantics clash with collapse determinism
\item
  no unified notion of ``truth event''
\end{itemize}

\subsubsection{4.3 VM-based isolates}\label{vm-based-isolates}

Rejected because:

\begin{itemize}
\tightlist
\item
  treat global state as external
\item
  incompatible with inversion semantics
\item
  no natural mapping to DAG lineage
\item
  higher resource overhead than SWS
\end{itemize}

\subsubsection{4.4 Single global world}\label{single-global-world}

Rejected because:

\begin{itemize}
\tightlist
\item
  totally breaks down under concurrency
\item
  no agent isolation
\item
  no speculative execution
\item
  no shadow semantics
\item
  no collapse concept
\end{itemize}

\begin{center}\rule{0.5\linewidth}{0.5pt}\end{center}

\subsection{5. Consequences}\label{consequences}

\subsubsection{5.1 Positive}\label{positive}

\begin{itemize}
\tightlist
\item
  deterministic concurrency
\item
  perfect isolation
\item
  unified model for humans + agents
\item
  reproducible compute
\item
  natural support for multi-agent workflows
\item
  seamless integration with DAG and WAL
\item
  elegant mapping to C\ensuremath{\Omega}MPUTER
\item
  safe speculative execution
\item
  easy rollback / discard semantics
\end{itemize}

\begin{center}\rule{0.5\linewidth}{0.5pt}\end{center}

\subsubsection{5.2 Negative}\label{negative}

\begin{itemize}
\tightlist
\item
  requires a more sophisticated kernel
\item
  collapse logic must be strong
\item
  debugging must support shadows
\item
  mental model is new for developers
\item
  requires real tooling to visualize
\end{itemize}

\begin{center}\rule{0.5\linewidth}{0.5pt}\end{center}

\subsubsection{5.3 Required Follow-Ups}\label{required-follow-ups}

This ADR now mandates:

\begin{itemize}
\tightlist
\item
  Section 2 of Architecture Doc
\item
  SWS lifecycle diagrams
\item
  Shadow memory model in Section 6 (ADR-0019 interplay)
\item
  Security model for SWS ownership (ADR-0017, ADR-0020)
\item
  Collapse semantics (ADR-0004)
\item
  Integration with Message Plane (RFC-0008)
\end{itemize}

\begin{center}\rule{0.5\linewidth}{0.5pt}\end{center}

\subsection{6. Decision}\label{decision-1}

\textbf{Accepted.}

Shadow Working Sets are the formal and singular process abstraction of
JITOS.

All subsequent architecture design flows from this.
