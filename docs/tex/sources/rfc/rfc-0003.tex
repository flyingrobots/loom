\section{JIT RFC-0003 --- Shadow Working Set Semantics (SWS
v1.0)}\label{jit-rfc-0003-shadow-working-set-semantics-sws-v1.0}

\subsection{The Observer Model, Epistemic Isolation, and the Collapse
Operator}\label{the-observer-model-epistemic-isolation-and-the-collapse-operator}

\begin{center}\rule{0.5\linewidth}{0.5pt}\end{center}

\begin{enumerate}
\def\labelenumi{\arabic{enumi}.}
\tightlist
\item
  Summary
\end{enumerate}

This RFC defines the semantics of Shadow Working Sets (SWS) --- the
foundational abstraction for:

\begin{itemize}
\tightlist
\item
  agent-level processes
\item
  parallel worldlines
\item
  speculative computation
\item
  observer-relative state
\item
  distributed editing
\item
  LLM/autonomous agent collaboration
\item
  safe local mutation
\item
  deterministic collapse into the global DAG
\end{itemize}

Shadow Working Sets represent local, subjective views of the universe
--- projections of the causal DAG --- which remain isolated until
resolved through a commit collapse event.

SWS is to JIT what ``process'' was to Unix --- except properly grounded
in physics, causality, and distributed truth.

\begin{center}\rule{0.5\linewidth}{0.5pt}\end{center}

\subsection{2. Motivation}\label{motivation}

Modern computation is multi-agent, distributed, asynchronous, and
subject to the limits of:

\begin{itemize}
\tightlist
\item
  latency
\item
  CAP theorem
\item
  causality
\item
  local knowledge
\item
  observer frames
\end{itemize}

Traditional operating systems expose a global mutable filesystem,
creating illusions of simultaneity and absolute state that are neither
true nor safe.

JIT acknowledges physical and epistemic reality:

No agent ever sees the whole universe. All views are subjective. Reality
is append-only.

Shadow Working Sets formalize this truth as a computational primitive.

\begin{center}\rule{0.5\linewidth}{0.5pt}\end{center}

\begin{enumerate}
\def\labelenumi{\arabic{enumi}.}
\setcounter{enumi}{2}
\tightlist
\item
  Definition
\end{enumerate}

A Shadow Working Set (SWS) is a temporary, isolated, agent-relative
projection of the causal DAG.

It includes:

\begin{lstlisting}
ShadowSet {
    id: uuid
    base_ref: string
    base_node: NodeID
    overlay_nodes: list<NodeID>
    virtual_tree_index: map<Path, NodeID>
    metadata: map<string,string>
}
\end{lstlisting}

\begin{itemize}
\tightlist
\item
  \passthrough{\lstinline!base\_node!} = the snapshot node the Shadow
  branches from
\item
  \passthrough{\lstinline!overlay\_nodes!} = ephemeral nodes
  representing edits
\item
  \passthrough{\lstinline!virtual\_tree\_index!} = projected filesystem
  view for the agent
\end{itemize}

All fields MUST be maintained in LMDB or equivalent.

\begin{center}\rule{0.5\linewidth}{0.5pt}\end{center}

\subsection{4. Formal Semantics}\label{formal-semantics}

Below are the rules governing SWS behavior.

\begin{center}\rule{0.5\linewidth}{0.5pt}\end{center}

\subsubsection{4.1 SWS Isolated Worldline}\label{sws-isolated-worldline}

An SWS is an epistemically isolated worldline which MUST NOT affect the
global DAG until committed.

\begin{itemize}
\tightlist
\item
  overlays exist only within the SWS
\item
  overlays cannot mutate the past
\item
  overlays cannot commit without collapsing
\item
  SWS cannot introduce new parents into the DAG
\end{itemize}

\begin{center}\rule{0.5\linewidth}{0.5pt}\end{center}

\subsubsection{4.2 SWS Causal Anchoring}\label{sws-causal-anchoring}

Each SWS MUST anchor to a single base node.

This ensures:

\begin{itemize}
\tightlist
\item
  deterministic diffs
\item
  deterministic rewrite potential
\item
  clear causal ancestry
\item
  unambiguous commit behavior
\end{itemize}

The base node MUST be a valid snapshot node.

\begin{center}\rule{0.5\linewidth}{0.5pt}\end{center}

\subsubsection{4.3 Local Consistency}\label{local-consistency}

Within an SWS:

\begin{itemize}
\tightlist
\item
  all edits must form a locally consistent graph
\item
  overlay nodes MUST obey RFC-0001 and RFC-0002 invariants
\item
  ordering must be deterministic within the SWS
\end{itemize}

Overlay graph is an internal micro-DAG, not merged with the global DAG.

\begin{center}\rule{0.5\linewidth}{0.5pt}\end{center}

\subsubsection{4.4 No Cross-SWS
Interference}\label{no-cross-sws-interference}

Two SWS MUST NOT observe each other's edits unless explicitly merged.

This enforces true agent isolation.

Agents operate in separate worldlines until collapse.

\begin{center}\rule{0.5\linewidth}{0.5pt}\end{center}

\subsubsection{4.5 Subjective Filesystem
Projection}\label{subjective-filesystem-projection}

SWS must maintain a virtual tree index for filesystem-like interactions.

This projection MUST be deterministic:

\begin{lstlisting}
virtual_tree_index[path] = most recent overlay node OR inherited from base snapshot
\end{lstlisting}

No real files need exist; this is an in-memory projection.

\begin{center}\rule{0.5\linewidth}{0.5pt}\end{center}

\subsubsection{4.6 Allowed Operations}\label{allowed-operations}

Operations permitted inside an SWS:

\begin{itemize}
\tightlist
\item
  patch application
\item
  diff computation
\item
  linting
\item
  refactoring
\item
  build computations
\item
  semantic agents (LLMs) modifying content
\item
  structural rewrites
\item
  preview merges
\item
  pattern-matches for rewrites
\end{itemize}

These operations MUST NOT modify the global DAG.

\begin{center}\rule{0.5\linewidth}{0.5pt}\end{center}

\subsection{5. Collapse Operator
(Commit)}\label{collapse-operator-commit}

The commit is the most important concept:

Commit collapses a subjectively computed SWS into an objectively real
DAG event.

The collapse operator:

\begin{lstlisting}
collapse(sws, global_dag) $\rightarrow$ new_snapshot_node
\end{lstlisting}

Collapse MUST obey:

\begin{itemize}
\tightlist
\item
  determinism
\item
  reproducibility
\item
  inversion semantics (via Inversion Engine)
\item
  causal ordering
\item
  full validation of overlays
\item
  merge conflict resolution
\item
  topological soundness
\end{itemize}

Collapse also MUST:

\begin{itemize}
\tightlist
\item
  generate a new snapshot node
\item
  discard the SWS afterward
\item
  update the ref target
\item
  log the event to WAL
\item
  advance the universe
\end{itemize}

SWS death is required:

Once committed, a Shadow can no longer exist. Potential becomes reality.
Superposition collapses into truth.

\begin{center}\rule{0.5\linewidth}{0.5pt}\end{center}

\subsection{6. Merge \& Conflict
Semantics}\label{merge-conflict-semantics}

If SWS collapse encounters divergence from
\passthrough{\lstinline!base\_node!}:

\begin{itemize}
\tightlist
\item
  the Inversion Engine MUST integrate changes
\item
  conflicts MUST be represented as multi-stage entries
\item
  merge MUST be deterministic
\item
  if unresolvable, commit MUST fail
\end{itemize}

Conflicts do not mutate the global DAG. They exist only during collapse
evaluation.

\begin{center}\rule{0.5\linewidth}{0.5pt}\end{center}

\subsection{7. Parallel Shadows}\label{parallel-shadows}

Multiple SWS may exist atop the same base\_node or different base nodes.

Rules:

\begin{itemize}
\tightlist
\item
  parallel SWS are independent
\item
  commit order provides serialization
\item
  later SWS commit must resolve against updated reality
\item
  deterministic merge rules must apply
\end{itemize}

This captures:

\begin{itemize}
\tightlist
\item
  concurrency
\item
  distributed editing
\item
  eventual consistency through collapse
\end{itemize}

\begin{center}\rule{0.5\linewidth}{0.5pt}\end{center}

\subsection{8. Deletion \& Failure
Semantics}\label{deletion-failure-semantics}

SWS MAY be:

\begin{itemize}
\tightlist
\item
  discarded by agent
\item
  timed out
\item
  destroyed by system
\item
  invalidated by upstream commits
\end{itemize}

Invalidation MUST occur if:

\begin{itemize}
\tightlist
\item
  \passthrough{\lstinline!base\_node!} no longer matches HEAD (or ref
  target)
\item
  collapse produces inconsistency
\item
  SWS violates invariants
\end{itemize}

In all cases:

Discarding an SWS must not affect the global DAG.

\begin{center}\rule{0.5\linewidth}{0.5pt}\end{center}

\subsection{9. Lifecycle}\label{lifecycle}

\subsubsection{I. Creation}\label{i.-creation}

\begin{lstlisting}
shadow.create(ref)
\end{lstlisting}

Creates new SWS anchored to ref's snapshot.

\subsubsection{II. Mutation}\label{ii.-mutation}

Edits accumulate in overlay nodes.

\subsubsection{III. Computation}\label{iii.-computation}

Agents synthesize new states.

\subsubsection{IV. Collapse}\label{iv.-collapse}

\begin{lstlisting}
shadow.commit(id)
\end{lstlisting}

Converts overlays $\rightarrow$ new snapshot.

\subsubsection{V. Death}\label{v.-death}

SWS is destroyed. Truth is updated.

\begin{center}\rule{0.5\linewidth}{0.5pt}\end{center}

\subsection{10. Security \& Isolation}\label{security-isolation}

\begin{itemize}
\tightlist
\item
  SWS cannot tamper with DAG
\item
  cannot impersonate global nodes
\item
  cannot bypass collapse
\item
  collapse validates signature/authorship
\item
  each SWS owner is tracked in metadata
\end{itemize}

\begin{center}\rule{0.5\linewidth}{0.5pt}\end{center}

\subsection{11. Why SWS Is Foundational}\label{why-sws-is-foundational}

SWS solves:

\begin{itemize}
\tightlist
\item
  distributed agent concurrency
\item
  local-only mutation
\item
  parallel universes
\item
  speculative worlds
\item
  conflict-free workflows
\item
  deterministic collapse
\item
  reproducible builds
\item
  multi-agent coordination
\item
  physics-consistent computation
\end{itemize}

Without SWS, JIT would revert to Git-like semantics and lose the
physics.

This RFC defines the heart of the inversion computation model.

\begin{center}\rule{0.5\linewidth}{0.5pt}\end{center}

\subsection{12. Status \& Next Steps}\label{status-next-steps}

This RFC precedes:

\begin{itemize}
\tightlist
\item
  \href{./RFC-0004.md}{RFC-0004} (Materialized Head)
\item
  \href{./RFC-0005.md}{RFC-0005} (Inversion Engine Semantics)
\item
  \href{./RFC-0006.md}{RFC-0006} (WAL Format \& Replay)
\item
  \href{./RFC-0007.md}{RFC-0007} (JIT RPC API)
\item
  \href{./RFC-0008.md}{RFC-0008} (Message Plane Integration)
\end{itemize}

\begin{center}\rule{0.5\linewidth}{0.5pt}\end{center}

\section{\texorpdfstring{\textbf{C\ensuremath{\Omega}MPUTER {\textbullet}
JITOS}}{C\ensuremath{\Omega}MPUTER {\textbullet} JITOS}}\label{cux3c9mputer-jitos}

{\textcopyright} 2025 James Ross {\textbullet} \href{https://flyingrobots.dev}{Flying {\textbullet} Robots} All
Rights Reserved
