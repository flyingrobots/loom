\section{JIT RFC-0004}\label{jit-rfc-0004}

\subsection{Materialized Head: The Human Projection Layer (MHP
v1.0)}\label{materialized-head-the-human-projection-layer-mhp-v1.0}

\begin{center}\rule{0.5\linewidth}{0.5pt}\end{center}

\subsection{1. Summary}\label{summary}

This RFC defines Materialized Head (MH) --- JIT's representation of the
global graph as a human-friendly filesystem. It is the illusion layer
that provides:

\begin{itemize}
\tightlist
\item
  familiar file I/O
\item
  editor/IDE integration
\item
  compatibility with existing tools
\item
  Git CLI support
\item
  deterministic file reconstruction
\end{itemize}

Materialized Head is not the true state of the universe --- it is a
shadow projection, analogous to Plato's Cave.

The DAG is reality. Materialized Head is perception.

\begin{center}\rule{0.5\linewidth}{0.5pt}\end{center}

\subsection{2. Motivation}\label{motivation}

Humans require:

\begin{itemize}
\tightlist
\item
  files
\item
  directories
\item
  text editors
\item
  terminal-based workflows
\item
  Git operations
\end{itemize}

But JIT uses:

\begin{itemize}
\tightlist
\item
  nodes
\item
  causal graphs
\item
  rewrites
\item
  SWS overlays
\item
  collapse events
\end{itemize}

There must be a bridge --- a projection system that maps a
high-dimensional causal geometry into a low-dimensional tree of files.

Materialized Head provides this layer without sacrificing:

\begin{itemize}
\tightlist
\item
  determinism
\item
  immutability
\item
  causal truth
\item
  JIT invariants
\end{itemize}

This is how humans work inside a post-file universe.

\begin{center}\rule{0.5\linewidth}{0.5pt}\end{center}

\subsection{3. Definition}\label{definition}

Materialized Head is a filesystem view backed by:

\begin{enumerate}
\def\labelenumi{\arabic{enumi}.}
\tightlist
\item
  Tree Index (LMDB)
\item
  Working Directory Mirror
\item
  Metadata Cache
\item
  Filesystem Watcher
\item
  Projection Engine
\end{enumerate}

It is tied to a ref (typically HEAD), representing one particular slice
of objective reality.

\begin{center}\rule{0.5\linewidth}{0.5pt}\end{center}

\subsection{4. Core Principles}\label{core-principles}

\subsubsection{4.1 MH is Epistemic}\label{mh-is-epistemic}

Materialized Head is not truth. It is a shadow of truth.

Only the DAG carries objective meaning. MH is a convenient fiction.

\begin{center}\rule{0.5\linewidth}{0.5pt}\end{center}

\subsubsection{4.2 MH Must Be
Deterministic}\label{mh-must-be-deterministic}

Given:

\begin{itemize}
\tightlist
\item
  a snapshot node
\item
  the node store
\item
  \href{./RFC-0001.md}{RFC-0001} encoding
\end{itemize}

Materialized Head \textbf{MUST} reconstruct the same file tree
everywhere.

\begin{center}\rule{0.5\linewidth}{0.5pt}\end{center}

\subsubsection{4.3 MH is Incremental}\label{mh-is-incremental}

Reconstruction \textbf{MUST} be incremental, not full:

\begin{itemize}
\tightlist
\item
  It \textbf{MUST} only materialize files that change.
\item
  Unchanged paths \textbf{MUST} remain untouched.
\item
  Directory structure \textbf{MUST} be updated lazily.
\end{itemize}

\begin{center}\rule{0.5\linewidth}{0.5pt}\end{center}

\subsubsection{4.4 MH Is Not
Write-Through}\label{mh-is-not-write-through}

Writes to the working directory NEVER mutate reality.

They:

\begin{itemize}
\tightlist
\item
  update MH state
\item
  trigger overlay creation
\item
  enter Shadow Working Sets (SWS)
\end{itemize}

MH is the input to SWS, not the substrate.

\begin{center}\rule{0.5\linewidth}{0.5pt}\end{center}

\subsubsection{4.5 MH Must Be Fast}\label{mh-must-be-fast}

Key performance requirements:

\begin{itemize}
\tightlist
\item
  git status \textless{} 20ms
\item
  file edits detected \textless{} 10ms
\item
  diff computation \textless{} 50ms
\end{itemize}

Achieved by:

\begin{itemize}
\tightlist
\item
  tree-index caching
\item
  file metadata hashing
\item
  fs watchers
\item
  minimal reconstruction
\end{itemize}

\begin{center}\rule{0.5\linewidth}{0.5pt}\end{center}

\subsection{5. Tree Index (The Real Data
Structure)}\label{tree-index-the-real-data-structure}

The Tree Index is the authoritative representation of the human-facing
file system.

It maps:

\begin{lstlisting}
path $\rightarrow$ FileProjection
\end{lstlisting}

Where:

\begin{lstlisting}
FileProjection {
node: NodeID
size: int
hash: blake3-256
mtime: timestamp
flags: bitset (conflict, executable, symlink, etc.)
}
\end{lstlisting}

Stored in LMDB for durability.

\begin{center}\rule{0.5\linewidth}{0.5pt}\end{center}

\subsection{6. Working Directory Mirror}\label{working-directory-mirror}

MH maintains a mirror of the file tree on disk under actual OS files.

Rules:

\begin{enumerate}
\def\labelenumi{\arabic{enumi}.}
\tightlist
\item
  Only modified files are rewritten.
\item
  Unchanged files are untouched between checkouts.
\item
  Conflicted files include conflict markers.
\item
  Deleted files are removed deterministically.
\item
  Symlinks are restored exactly.
\end{enumerate}

This mirror is the cave wall.

\begin{center}\rule{0.5\linewidth}{0.5pt}\end{center}

\subsection{7. Conflict Semantics}\label{conflict-semantics}

During merges/rebases/inversions:

Materialized Head \textbf{MUST} support multi-stage entries:

\begin{itemize}
\tightlist
\item
  \passthrough{\lstinline!BASE!}
\item
  \passthrough{\lstinline!OURS!}
\item
  \passthrough{\lstinline!THEIRS!}
\item
  \passthrough{\lstinline!RESOLVED!}
\end{itemize}

The index tracks all three, but the filesystem presents textual conflict
markers for human resolution.

Example markers (must match Git exactly):

\textless\textless\textless\textless\textless\textless\textless{} OURS
\ldots content\ldots{} ======= \ldots content\ldots{}
\textgreater\textgreater\textgreater\textgreater\textgreater\textgreater\textgreater{}
THEIRS

After git add, a new file-chunk node is created and the conflict entry
becomes \passthrough{\lstinline!RESOLVED!}.

\begin{center}\rule{0.5\linewidth}{0.5pt}\end{center}

\subsection{8. Relationship to SWS}\label{relationship-to-sws}

\begin{quote}
{[}!WARNING{]} \emph{This is critical:}
\end{quote}

\subsubsection{8.1 MH serves humans; SWS serves
agents.}\label{mh-serves-humans-sws-serves-agents.}

MH performs:

\begin{itemize}
\tightlist
\item
  file reads
\item
  file writes
\item
  Git CLI operations
\item
  editor interaction
\end{itemize}

SWS performs:

\begin{itemize}
\tightlist
\item
  ephemeral computation
\item
  speculative operations
\item
  agent-based edits
\item
  automated transformations
\item
  previews
\end{itemize}

\subsubsection{8.2 MH must remain consistent with DAG +
SWS}\label{mh-must-remain-consistent-with-dag-sws}

When:

\begin{itemize}
\tightlist
\item
  commit occurs
\item
  ref updates
\item
  remote sync
\item
  SWS collapse
\end{itemize}

MH must merge its view with the new universe state.

This may cause:

\begin{itemize}
\tightlist
\item
  local changes being rebased
\item
  conflicts
\item
  forced refresh
\item
  incremental reconstruction
\end{itemize}

Consistency \textbf{MUST} be maintained at all times.

\begin{center}\rule{0.5\linewidth}{0.5pt}\end{center}

\subsection{9. Filesystem Watcher}\label{filesystem-watcher}

MH \textbf{MUST} include a watcher that detects:

\begin{itemize}
\tightlist
\item
  file edits
\item
  renames
\item
  deletes
\item
  directory creation
\end{itemize}

Watcher events become SWS overlays.

MH does NOT ``write'' to the DAG directly --- only SWS can trigger
commit.

\begin{center}\rule{0.5\linewidth}{0.5pt}\end{center}

\begin{enumerate}
\def\labelenumi{\arabic{enumi}.}
\setcounter{enumi}{9}
\tightlist
\item
  Checkout Semantics
\end{enumerate}

During checkout:

\begin{enumerate}
\def\labelenumi{\arabic{enumi}.}
\tightlist
\item
  Retrieve snapshot node
\item
  Reconstruct tree-index
\item
  Apply incremental updates to working directory
\item
  Discard any out-of-date SWS associated with the ref
\item
  Update MH metadata
\item
  Update active shadow context
\end{enumerate}

Performance requirement:

\begin{itemize}
\tightlist
\item
  checkout \textless{} 150ms on typical repos
\end{itemize}

\begin{center}\rule{0.5\linewidth}{0.5pt}\end{center}

\subsection{11. Revert \& Reset Behavior}\label{revert-reset-behavior}

Revert:

\begin{itemize}
\tightlist
\item
  create inversion node
\item
  update MH via tree-index
\item
  rewrite filesystem for changed paths
\end{itemize}

Reset:

\begin{itemize}
\tightlist
\item
  detach MH from previous overlays
\item
  reconstruct tree-index
\item
  remove untracked files if requested
\end{itemize}

MH must remain deterministic.

\begin{center}\rule{0.5\linewidth}{0.5pt}\end{center}

\subsection{12. Sync Semantics (Remote)}\label{sync-semantics-remote}

When pulling from remote:

\begin{itemize}
\tightlist
\item
  fetch new snapshot nodes
\item
  merge with MH tree-index
\item
  handle conflicts
\item
  present conflict markers
\item
  ensure deterministic rehydration (RFC-0003)
\end{itemize}

MH must remain consistent with the updated universe.

\begin{center}\rule{0.5\linewidth}{0.5pt}\end{center}

\subsection{13. Failure \& Crash
Semantics}\label{failure-crash-semantics}

Upon crash or abrupt termination:

\begin{itemize}
\tightlist
\item
  MH reconstruction \textbf{MUST} be possible from:
\item
  the DAG
\item
  the WAL
\item
  the tree-index
\end{itemize}

MH is fully recoverable as long as the substrate exists.

Filesystem inconsistencies are ALWAYS repaired during initialization.

\begin{center}\rule{0.5\linewidth}{0.5pt}\end{center}

\subsection{14. Security}\label{security}

MH must obey:

\begin{itemize}
\tightlist
\item
  no SWS can bypass commit
\item
  no direct DAG mutation
\item
  no arbitrary node insertion
\item
  permissions match underlying OS
\item
  signature validations enforced at commit
\end{itemize}

Filesystem writes are sandboxed to working directory.

\begin{center}\rule{0.5\linewidth}{0.5pt}\end{center}

\subsection{15. Why MH Is Foundational}\label{why-mh-is-foundational}

Materialized Head provides:

\begin{itemize}
\tightlist
\item
  human compatibility
\item
  backwards Git compatibility
\item
  stable user workflows
\item
  deterministic projections
\item
  performance
\item
  reproducibility
\item
  separation of observer frames
\item
  the illusion needed for humans to operate in a post-file world
\end{itemize}

MH is the human lens through which the geometric universe of JIT becomes
visible.

It is the shadow on the wall.

\begin{center}\rule{0.5\linewidth}{0.5pt}\end{center}

\section{\texorpdfstring{\textbf{C\ensuremath{\Omega}MPUTER {\textbullet}
JITOS}}{C\ensuremath{\Omega}MPUTER {\textbullet} JITOS}}\label{cux3c9mputer-jitos}

{\textcopyright} 2025 James Ross {\textbullet} \href{https://flyingrobots.dev}{Flying {\textbullet} Robots} All
Rights Reserved
