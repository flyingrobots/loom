\sectionbreak
\section{Provenance Payloads and Computational Holography}
\label{sec:holography}

We now make precise the idea that the entire interior evolution of a
computation can be encoded on a ``boundary'': an initial state together
with a finite provenance payload.  This is the formal content of
\emph{computational holography}.

\subsection{Microsteps and derivation graphs}

Fix a rule set $R$ and tick semantics as in
Theorem~\ref{thm:tick-confluence}.  A \emph{microstep} is a single
scheduler tick whose batch contains exactly one skeleton DPOI step
(possibly accompanied by attachment steps in preserved fibers).  We
write
\[
  S_i \;\Rewrite^{\mu_i}\; S_{i+1}
\]
for such a microstep, where the label $\mu_i$ records:
\begin{itemize}[leftmargin=*]
  \item the rule identifier $p \in R$;
  \item the match identifier for the skeleton step;
  \item any attachment-rule identifiers used in the same tick;
  \item auxiliary metadata (timestamps, policy hashes, etc.).
\end{itemize}
We abstract this as a finite record in some fixed alphabet; in
particular, we assume it has a self-delimiting encoding.

For a value $v$ in some state $S_i$ we define a \emph{derivation graph}
$\mathcal{D}(v)$ whose nodes are intermediate values and whose edges are
microstep applications that produced them; the construction is standard
and we omit the routine details.  For a finite derivation
\[
  S_0 \Rewrite^{\mu_0} S_1 \Rewrite^{\mu_1} \cdots
  \Rewrite^{\mu_{n-1}} S_n,
\]
each microstep reads values in some $S_j$ and produces new values in
the immediately later state $S_{j+1}$, so every provenance edge in
$\mathcal{D}(v)$ points from a value in $S_j$ to a value in $S_{j+1}$
(hence tick indices strictly increase along edges).  Immutability
ensures that values are never updated in-place, only created at later
ticks.  Since each RMG state $S_j$ is finite and there are only $n+1$
such states along the derivation, $\mathcal{D}(v)$ has finitely many
nodes; and because tick indices strictly increase along edges, every
causal chain leading to $v$ has length at most $n$, so $\mathcal{D}(v)$
is a finite acyclic graph.

\subsection{AION state packets as an instance}

Operationally, the \AION{} runtime records each state transition as an
\emph{Aion State Packet} (ASP)
\[
  \alpha = (S_{\mathrm{in}}, S_{\mathrm{out}}, R, P, t, \sigma),
\]
where $S_{\mathrm{in}}$ and $S_{\mathrm{out}}$ are content-addressed
graph states, $R$ identifies the ruleset, $t$ is a Chronos index,
$\sigma$ is an integrity/authentication tag, and $P$ is a
\emph{provenance payload}.  In the formal development below, we work
with abstract microsteps and payloads; the ASP is one concrete
instantiation of this pattern.

\paragraph{Example (Toy \AION{} state packet).}
As a toy example, consider a computation that increments an integer.
The input state $S_{\mathrm{in}}$ contains a literal $x=5$, the output
state $S_{\mathrm{out}}$ contains both $x=5$ and a result $y=6$, the
ruleset $R$ includes a rule $\mathsf{inc}$ that adds one, and the
payload $P$ consists of a single microstep:
``read $x$, apply $+1$, write $y$''.  This entire evolution is recorded
as a single ASP
$\alpha = (S_{\mathrm{in}}, S_{\mathrm{out}}, R, P, t, \sigma)$.
This illustrates how a seemingly atomic step at the outer RMG layer
can internally encode many microsteps.

\begin{definition}[Provenance payload and wormhole]
Let $S_0$ be an initial RMG state.  A \emph{provenance payload} of
length $n$ is a sequence
\[
  P = (\mu_0,\mu_1,\dots,\mu_{n-1})
\]
of microstep labels such that, by determinism of the tick semantics,
there exists a unique (up to isomorphism) sequence of states
\[
  S_0 \Rewrite^{\mu_0} S_1 \Rewrite^{\mu_1} \cdots
       \Rewrite^{\mu_{n-1}} S_n
\]
obtained by applying the corresponding ticks under the scheduler.
We call the pair $(S_0,P)$ a \emph{wormhole}, reflecting that it
collapses an entire derivation into a single boundary edge.  Its
\emph{volume} is the derivation path $S_0 \Rewrite^\ast S_n$, and its
\emph{boundary} is the pair $(S_0,P)$.
\end{definition}

By tick-level confluence
(Theorem~\ref{thm:tick-confluence}), any interleaving of concurrent
matches compatible with $P$ yields a final state isomorphic to $S_n$.

\subsection{Backward provenance completeness}

We first show that provenance is complete in the backward direction:
every value admits a unique causal history inside the wormhole.

As a design constraint on the runtime, we assume:
\begin{itemize}[leftmargin=*]
  \item \emph{No re-derivation (single producer):} if a microstep
    would produce a value whose content hash already appears in any
    stored state, the runtime reuses the existing value instead of
    recording a new producing microstep.  Thus every content-addressed
    value has a unique producing microstep in the ledger prefix.
\end{itemize}

\begin{theorem}[Backward provenance completeness]\label{thm:backward}
Let $(S_0,P)$ be a wormhole with volume
$S_0 \Rewrite^{\mu_0} \cdots \Rewrite^{\mu_{n-1}} S_n$, and let
$v^\ast$ be a value occurring in $S_n$.  Under the assumptions of
total provenance capture, immutable content-addressed values, and the
no re-derivation (single-producer) property, the derivation graph
$\mathcal{D}(v^\ast)$ inside this wormhole is unique up to
isomorphism.
\end{theorem}

\begin{proof}
We proceed by induction on the depth of $v^\ast$ in the derivation
graph.

By total provenance capture, every value in any $S_i$ is either present
in the initial state $S_0$ or produced as the output of some microstep
labelled by $\mu_j$ whose inputs are values in an earlier state $S_j$.
By immutability and content addressing, equal hashes coincide with
equal values; there is no aliasing of distinct values.

Define the depth of a value to be the length of the longest path in
$\mathcal{D}(v)$ from a root (an initial literal in $S_0$) to $v$.
Because each microstep increases depth by at most one and the payload
is finite, every $v$ has finite depth.

\emph{Base case.}
If $v^\ast$ has depth $0$, then it is a literal occurring already in
$S_0$; its derivation graph consists of a single node and is therefore
unique.

\emph{Inductive step.}
Assume the statement holds for all values of depth at most $k$, and let
$v^\ast$ have depth $k+1$.  By total provenance capture, $v^\ast$ is
the output of at least one microstep in $P$.  By the no re-derivation
(single-producer) assumption, there is in fact a unique producing
microstep $\mu_j$ with output position $o$, applied at state $S_j$;
its inputs $v_1,\dots,v_m$ are values in $S_j$ of depth at most $k$.
By the induction hypothesis, each $\mathcal{D}(v_i)$ is unique up to
isomorphism.  Determinism of the tick semantics ensures that the
microstep $\mu_j$ applied to these inputs produces $v^\ast$ uniquely.

Thus $\mathcal{D}(v^\ast)$ is obtained by gluing together the unique
graphs $\mathcal{D}(v_i)$ along the unique producing microstep
$\mu_j$.  Any alternative derivation graph for $v^\ast$ would either
change the last producing microstep or some subgraph $\mathcal{D}(v_i)$.
The former is ruled out by the single-producer assumption, and the
latter by the induction hypothesis.  Hence $\mathcal{D}(v^\ast)$ is
unique up to isomorphism.
\end{proof}

\subsection{Computational holography}

The key insight is that the payload $P$ carries precisely the
information required to reconstruct the entire interior evolution.
We now formalise the ``boundary encodes volume'' slogan.

\begin{definition}[Reconstruction procedure]
Given a wormhole $(S_0,P)$ with
$P = (\mu_0,\dots,\mu_{n-1})$, define
\[
  \Recon(S_0,P) = (S_0,S_1,\dots,S_n)
\]
by the recursion
\[
  S_{i+1} \;=\; \Apply(S_i,\mu_i)
\]
for $0 \le i < n$, where $\Apply$ executes the unique microstep
described by~$\mu_i$ under the deterministic tick semantics.
Determinism ensures that each $S_{i+1}$ is uniquely determined (up to
isomorphism).  Furthermore, tick-level confluence
(Theorem~\ref{thm:tick-confluence}) guarantees that any internal
interleaving of concurrent matches compatible with $\mu_i$ yields an
isomorphic successor.
\end{definition}

\begin{theorem}[Computational holography]
\label{thm:holography}
Let $(S_0,P)$ be a wormhole.  Then:
\begin{enumerate}[leftmargin=*]
  \item The reconstruction procedure terminates and produces a unique
    (up to isomorphism) volume $S_0 \Rewrite^\ast S_n$.
  \item Conversely, any finite derivation
    $S_0 \Rewrite^{\mu_0} \cdots \Rewrite^{\mu_{n-1}} S_n$
    induces a wormhole $(S_0,P)$ with $P=(\mu_0,\dots,\mu_{n-1})$
    whose reconstruction yields an isomorphic volume.
\end{enumerate}
Thus the boundary data $(S_0,P)$ is information-complete with respect
to the interior evolution: up to isomorphism, each finite derivation
volume corresponds to a unique boundary and vice versa.  In particular,
boundaries (considered up to isomorphism of $S_0$) are in bijection
with isomorphism classes of finite derivation volumes.
\end{theorem}

\begin{figure}[t]
  \centering
  \begin{tikzpicture}[
      state/.style={rectangle,draw=blue!70!black,fill=blue!5,thick,minimum width=8mm,
                    minimum height=6mm,inner sep=1pt},
      payload/.style={rectangle,draw=green!60!black,fill=green!5,thick,rounded corners,
                      minimum width=35mm,minimum height=16mm,
                      align=center},
      arrow/.style={-Latex,thick,blue!70!black},
      brace/.style={decorate,decoration={brace,amplitude=5pt}},
      >=Latex
    ]

    % Volume side
    \node at (0,2.0) {\small \textbf{Interior evolution (volume)}};

    \node[state] (S0) at (0,1.0) {$S_0$};
    \node[state] (S1) at (1.5,1.0) {$S_1$};
    \node[state] (S2) at (3.0,1.0) {$S_2$};
    \node            (dots) at (4.2,1.0) {$\cdots$};
    \node[state] (Sn) at (5.4,1.0) {$S_n$};

    \draw[arrow] (S0) -- node[above,sloped]{\scriptsize $\mu_0$} (S1);
    \draw[arrow] (S1) -- node[above,sloped]{\scriptsize $\mu_1$} (S2);
    \draw[arrow] (S2) -- node[above,sloped]{\scriptsize $\mu_2$} (dots);
    \draw[arrow] (dots) -- node[above,sloped]{\scriptsize $\mu_{n-1}$} (Sn);

    \draw[brace,thick] ([yshift=-4pt]S0.south west) --
                       ([yshift=-4pt]Sn.south east)
       node[midway,below=6pt]{\scriptsize explicit derivation path};

    % Boundary side
    \node at (0,-0.4) {\small \textbf{Boundary encoding}};

    \node[payload] (Boundary) at (3.0,-1.3)
      {%
      initial state: $S_0$\\[-2pt]%
      \rule{30mm}{0.3pt}\\[-2pt]%
      payload: $P=(\mu_0,\dots,\mu_{n-1})$%
      };

    \draw[brace,thick] ([yshift=-4pt]Boundary.south west) --
                       ([yshift=-4pt]Boundary.south east)
      node[midway,below=6pt]{\scriptsize stored as a single edge label};

    % Mapping arrow
    \draw[<->,arrow] (Sn.south) .. controls +(0,-0.9) and +(0,0.9) ..
                     (Boundary.north)
      node[midway,right=2pt,align=left,inner sep=1pt]
        {\scriptsize computational\\[-1pt] \scriptsize holography};

  \end{tikzpicture}
  \caption{Computational holography: the full interior evolution
  $S_0 \Rewrite S_1 \Rewrite \cdots \Rewrite S_n$ (volume) is uniquely
  reconstructible from the boundary data $(S_0,P)$, where $P$ is the
  provenance payload attached to a \textbf{single RMG edge}.}
  \label{fig:holography-volume-boundary}
\end{figure}

\begin{proof}
(1) Termination is immediate from the finiteness of $P$: the recursion
defining $\Recon(S_0,P)$ executes exactly $n$ steps.  At each step,
$\Apply$ is defined because $\mu_i$ was assumed to be a valid
microstep label; determinism and tick-level confluence guarantee that
the resulting state $S_{i+1}$ is unique up to isomorphism.

(2) Given a finite derivation as in the statement, simply take
$P=(\mu_0,\dots,\mu_{n-1})$.  The reconstruction procedure follows
exactly the same sequence of microsteps from $S_0$; by determinism,
the reconstructed states are isomorphic to the original ones at each
index, hence the reconstructed volume is isomorphic to the given
volume.

The two directions together induce a bijection between isomorphism
classes of finite derivations and isomorphism classes of boundaries
$(S_0,P)$, where boundaries are quotiented by isomorphism of the
initial state $S_0$.
\end{proof}

\begin{remark}[Relation to physical holographic principles]
\label{rem:physical-holography}
The term ``holography'' is used in several distinct communities.  In
high-energy physics, the holographic principle---most prominently
realised in the AdS/CFT correspondence---asserts that the information
content of a bulk spacetime region can be encoded on a lower-dimensional
boundary.  Our use of ``computational holography'' is a precise,
information-theoretic analogue in a discrete, deterministic setting:
the ``volume'' is the interior derivation sequence
$S_0 \Rewrite S_1 \Rewrite \cdots \Rewrite S_n$, and the ``boundary''
is the pair $(S_0,P)$, where $P$ is a provenance payload.
Theorem~\ref{thm:holography} establishes that this boundary is
information-complete with respect to the volume in the sense of
algorithmic reconstruction.  We do not assume any geometric or
quantum-mechanical structure, though it is tempting to speculate about
future connections.
\end{remark}

Thus the entire ``volume'' of the computation is encoded on the
boundary.  We will refer to this property as computational holography.
