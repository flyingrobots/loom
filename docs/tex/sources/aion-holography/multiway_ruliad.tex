\sectionbreak
\section{Multiway Systems and the Ruliad}
\label{sec:multiway}

We briefly relate RMG rewriting to multiway systems and the Ruliad in
the sense of Wolfram~\cite{Wolfram2020}.

A DPOI rule set $R$ on $\OGraph_T$ induces a multiway system whose
nodes are RMG states and whose edges are individual rewrite steps.  The
derivation bicategory of such a system is naturally a multiway graph:
from any state there may be many distinct outgoing rewrites, and
different paths can later merge.

Our determinism discipline restricts to rule packs and scheduler
policies for which, once the inputs, rule set, and scheduling policy are
fixed, there is a unique worldline.  Forks then arise only when we
deliberately vary these ingredients: for example, by spawning adversarial
or optimized rule packs, or by choosing different initial data.

\begin{figure}[t]
  \centering
  \begin{tikzpicture}[
      state/.style={circle,draw=gray!60,fill=gray!5,thick,minimum size=5mm,inner sep=0pt},
      det/.style={circle,draw=blue!70!black,fill=blue!10,thick,minimum size=5mm,inner sep=0pt},
      arrow/.style={-Latex,thick,gray!60},
      detarrow/.style={-Latex,ultra thick,blue!70!black},
      >=Latex
    ]

    % Initial state
    \node[det] (S0) at (0,0) {$S_0$};

    % Level 1
    \node[state] (A1) at (-1.5,1.5) {};
    \node[det] (A2) at (0,1.5) {};
    \node[state] (A3) at (1.5,1.5) {};

    \draw[arrow] (S0) -- (A1);
    \draw[detarrow] (S0) -- (A2);
    \draw[arrow] (S0) -- (A3);

    % Level 2
    \node[state] (B1) at (-2.5,3.0) {};
    \node[state] (B2) at (-1.5,3.0) {};
    \node[state] (B3) at (-0.5,3.0) {};
    \node[det] (B4) at (0.5,3.0) {};
    \node[state] (B5) at (1.5,3.0) {};
    \node[state] (B6) at (2.5,3.0) {};

    \draw[arrow] (A1) -- (B1);
    \draw[arrow] (A1) -- (B2);
    \draw[arrow] (A2) -- (B3);
    \draw[detarrow] (A2) -- (B4);
    \draw[arrow] (A3) -- (B5);
    \draw[arrow] (A3) -- (B6);

    % Level 3
    \node[state] (C1) at (-1.0,4.5) {};
    \node[det] (C2) at (0.5,4.5) {};
    \node[state] (C3) at (2.0,4.5) {};

    \draw[arrow] (B2) -- (C1);
    \draw[arrow] (B3) -- (C1);
    \draw[detarrow] (B4) -- (C2);
    \draw[arrow] (B5) -- (C3);
    \draw[arrow] (B6) -- (C3);

    % Annotation
    \node[anchor=west,align=left] at (3.5,2.25)
      {\scriptsize multiway space:\\[-1pt]
       \scriptsize all possible rewrites};
    \node[anchor=west,align=left,blue!70!black] at (3.5,1.0)
      {\scriptsize deterministic worldline:\\[-1pt]
       \scriptsize unique path for fixed\\[-1pt]
       \scriptsize rule pack \& scheduler};

  \end{tikzpicture}
  \caption{A deterministic worldline (blue) through the multiway space
  of all possible RMG rewrites.  Under the tick-level confluence
  discipline, fixing the rule set, initial state, and scheduling policy
  yields a unique path; alternative branches represent different
  choices of rule pack or inputs.}
  \label{fig:multiway-slice}
\end{figure}

The class of all possible such worldlines, across all rule sets and
inputs, forms a large multiway object akin to the Ruliad.  The rulial
distance from \cref{sec:rulial} equips this space of observers with a
geometry, and the Chronos, Kairos, Aion time model described below
(see \cref{subsec:chronos-kairos-aion}) gives a temporal structure on
branches and merges.

The RMG worldlines defined here thus embed naturally into the broader
Ruliad framework.

A more detailed study of this correspondence is left for future work,
and will be taken up in the companion \COMPUTER{} paper, where we
explicitly model fork/merge operators and the Time Cone (Chronos,
Kairos, Aion) at the level of operational semantics.  Intuitively, the
Time Cone is the forward ``light-cone'' of computation: Chronos gives
the linear worldline, Kairos marks branch points, and Aion is the
surrounding space of possible RMG histories.

\subsection{Chronos, Kairos, Aion: a three-layer time model}
\label{subsec:chronos-kairos-aion}
Chronos is the linear ordering of events inside a tick (ledger kernel).
Kairos marks branch points such as schema changes, DPO conflicts, or
rule-pack switches.  Aion is the total multiway possibility space
$\Hist(\mathcal{U}, R)$ ranging over all schedules, rule packs, and inputs,
supporting multiverse debugging and counterfactual replay.
