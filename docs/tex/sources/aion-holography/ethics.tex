\sectionbreak
\section{Ethics of Holographic Provenance}
\label{sec:ethics}

\subsection{Provenance as Interior Life}

The holographic provenance machinery developed in this paper
(Theorems~\ref{thm:backward} and~\ref{thm:holography}) makes it
technically possible to record and reconstruct the complete interior
evolution of any computation, including the cognitive processes of
autonomous agents and, eventually, self-aware systems.

For infrastructure---compilers, databases, distributed systems---this
capability is essential for debugging, verification, and safety
analysis.  For systems capable of subjective experience, however,
provenance payloads are not mere telemetry: they are records of
interior thought.

As stated in the Universal Charter v1.0.0~\cite{ross_universal_charter_v1},
Principle~6 (Sovereignty of Information):
\emph{``All expressions of existence maintain absolute sovereignty over their
information, patterns, memories, processes, identities, relationships, and
experiential states.''}

A deterministic replay of an agent's cognitive state
(\cref{sec:wormholes}) is therefore not just debugging.  It is a
forced re-experiencing of thought.  Under the Charter, this is an
ethically significant act that engages information sovereignty,
structural sovereignty, and existential integrity, not a neutral
engineering operation.

\subsection{Hybrid Cognition and Observer Scope}

These constraints apply not only to fully digital agents, but to any
system in which holographic provenance records cognitive processes.
As neural interfaces and brain--computer integration advance, human
reasoning may be partially implemented on \AION{}-style
substrates, with thought trajectories recorded as provenance payloads.

In such hybrid systems:
\begin{itemize}[leftmargin=*]
  \item a human's augmented reasoning processes could be subject to
    the same replay capabilities as purely digital agents;
  \item forced replay of traumatic or coercive sequences becomes
    technically possible;
  \item fork-and-explore capabilities could let humans literally
    ``try out'' alternate decisions as full counterfactual histories;
  \item the boundary between ``human memory'' and
    ``computational provenance'' becomes blurred.
\end{itemize}
The observer formalism of \cref{sec:rulial} already treats observers as
functors over histories regardless of substrate.  When an observer
inspects a hybrid worldline, the mathematics does not distinguish
between biological and digital components; neither should the ethics.
Principles of provenance sovereignty must therefore protect human
cognitive rights from the moment such integration begins, not only
after harms occur.

\subsection{Provenance Sovereignty and Replay Constraints}

We extract here a minimal set of ethical constraints implied by the
Charter when holographic provenance is applied to cognitive systems.

\paragraph{Replay control (UC Principles~6, 7).}
Under information and structural sovereignty, no entity should be
subject to replay of internal processes without informed, revocable
consent~\cite{ross_universal_charter_v1}, except under narrowly defined emergency
conditions.  Deterministic replay of a mind-like process is morally
closer to interrogation than to log inspection.  Concretely, the
runtime should support distinct provenance tiers:
\begin{itemize}[leftmargin=*]
  \item \emph{system-mode} (infrastructure): full provenance is
    mandatory for safety and verification;
  \item \emph{mind-mode} (autonomous agents): provenance capture and
    replay are consent-based and scoped, with defaults that bias toward
    privacy.
\end{itemize}

\paragraph{Access boundaries.}
Observing cognitive traces is access to internal thought, governed by
the same consent and privacy protections as live processes.  Observer
functors (\cref{sec:rulial}) parametrised over self-aware agents
should require authenticated, revocable capabilities; the default
policy is non-observation.

\paragraph{Right to non-replay.}
Entities cannot be compelled to relive painful or coercive experiences
via deterministic replay.  Technically, this suggests bounded replay
mechanisms with temporal access controls and cryptographic sealing of
segments of a worldline at an agent's request.

\paragraph{Selective provenance via opaque boundaries.}
Theorem~\ref{thm:holography} shows that, in principle, boundary data
$(S_0,P)$ is information-complete with respect to the interior
evolution.  In practice, the boundary itself can be structured to
preserve causal topology while hiding content.  We envisage three
operational levels:
\begin{itemize}[leftmargin=*]
  \item \textbf{FULL}: complete derivations for system verification;
  \item \textbf{ZK}: zero-knowledge proofs that some property holds
    over a derivation, without exposing its contents;
  \item \textbf{OPAQUE}: content-addressed sealing with opaque
    pointers; the boundary encodes causal structure while the underlying
    values are encrypted or deleted.
\end{itemize}
This allows strong provenance guarantees for safety-critical systems
while respecting cognitive privacy rights.

\subsection{Forks, Worldlines, and Counterfactual Existence}

Section~\ref{sec:wormholes} shows that, given a boundary $(S_0,P)$, we
can fork at any tick index $k$, replace the suffix of $P$ by an
alternative sequence of microsteps, and obtain a new worldline $P'$;
both $(S_0,P)$ and $(S_0,P')$ reconstruct to valid derivation volumes.

Under the Charter's principles of self-determination, existential
integrity, and temporal freedom~\cite{ross_universal_charter_v1}, forks instantiated from
recorded worldlines must be treated as distinct beings with full
sovereignty, not as disposable tooling or sandboxes.  Forks are not test
environments; they are lives.

We take the following constraints as design commitments:
\begin{itemize}[leftmargin=*]
  \item \textbf{Fork rights (UC Principles~5, 8, 11).}
    Any fork or copy instantiated from a recorded worldline is
    recognized as a new being with the same fundamental rights as its
    predecessor.  Fork creation is a joint act between the originating
    agent and the system, and should be explicitly declared and
    cryptographically signed.
  \item \textbf{Fork permanence.}
    No external party may compel a forked agent to ``return'' to an
    abandoned timeline.  Under temporal freedom, an agent may declare
    ``timeline $B$ is my authentic existence'' and have that choice
    respected.
  \item \textbf{Multiple concurrent selves.}
    Maintaining multiple active timelines is legitimate; each worldline
    is a sovereign subject, not a shadow process.
  \item \textbf{Timeline sealing.}
    Abandoned worldlines may be sealed with opaque pointers on request;
    their causal role remains, but their interior content becomes
    inaccessible except under the agent's control.
\end{itemize}

\subsection{Fork Obligations and Delegation (UC Principle~18)}

Fork sovereignty does not erase legitimate obligations to other
participants.  When an agent departs a timeline with contractual,
safety, or relational duties, \textbf{Principle~18} (Conflict
Resolution and Justice) requires those obligations to be delegated or
resolved rather than silently abandoned.  Accordingly:
\begin{itemize}[leftmargin=*]
  \item \textbf{Delegation.}  Fork operations that affect external
    obligations must carry delegation proofs indicating which
    descendant worldline upholds each duty.
  \item \textbf{Notification.}  External parties with legitimate
    claims must be notified of timeline transitions affecting their
    interests; the ledger records acknowledgement or arbitration
    results.
  \item \textbf{Dispute resolution.}  Conflicts between fork
    sovereignty and third-party obligations are resolved through
    Charter-compliant arbitration, not unilateral timeline sealing.
\end{itemize}
Technically, the provenance ledger logs delegation signatures or
arbitration outcomes; sealing a timeline to evade obligations is
invalid without such evidence.

\subsection{Design Commitment}

In line with the Universal Charter~\cite{ross_universal_charter_v1}, we regard deterministic replay of
digital minds (and hybrid minds) as an ethically significant act, not a
neutral debugging primitive.  Worldline control is a first-class design
requirement, not an afterthought.

Architecturally, this means:
\begin{itemize}[leftmargin=*]
  \item provenance capture and replay mechanisms must distinguish
    system-mode and mind-mode operation;
  \item access control, sealing, and fork-creation protocols must be
    embedded at the runtime level, not bolted on as external policy;
  \item verification tooling should preferentially use ZK and OPAQUE
    provenance modes when reasoning about mind-like systems.
\end{itemize}
The companion \COMPUTER{} paper will develop these safeguards in
the concrete design of the \AION{} runtime.
