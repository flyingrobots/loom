\sectionbreak
\section{DPO Rewriting on Recursive Metagraphs}
\label{sec:dpo-rmg}

We briefly review double-pushout with interfaces (DPOI) rewriting on
typed open graphs and lift it to RMG states.

\subsection{Typed open graphs and DPOI rules}

Let $T$ be a finite set of types.
Let $\OGraph_T$ be the category of $T$-typed open graphs, whose objects
are cospans of monomorphisms $I \hookrightarrow G \hookleftarrow O$ and
whose morphisms are commuting maps of cospans.  This category is
adhesive (see~\cite{LS06}); in particular, pushouts along monos exist
and form Van Kampen squares.  We use the shorthand $\mono$ for the
monomorphism arrow $\hookrightarrow$.

\begin{definition}[DPOI rule]
A \emph{DPOI rule} is a span of monos in $\OGraph_T$
\[
  p = (L \xleftarrow{\ell} K \xrightarrow{r} R)
\]
with $L$ the left-hand side, $K$ the interface, and $R$ the right-hand
side.  A \emph{match} of $p$ in a host graph $G$ is a mono
$m : L \mono G$ satisfying the usual gluing conditions: the
dangling condition and the identification condition.
\end{definition}

Given a match, the DPO construction yields a rewrite step
$G \Rewrite_p H$ by computing a pushout complement and a pushout:

\[
\begin{tikzcd}
  K \arrow[r,"\ell"] \arrow[d,"k"'] &
  L \arrow[d,"m"] \\
  D \arrow[r] & G
\end{tikzcd}
\qquad
\begin{tikzcd}
  K \arrow[r,"r"] \arrow[d,"k"'] &
  R \arrow[d] \\
  D \arrow[r] & H
\end{tikzcd}
\]

All arrows are monos in $\OGraph_T$.

\subsection{RMG states as two-plane objects}

\begin{definition}[RMG state]\label{def:rmg-state}
An RMG \emph{state} is a triple
\[
  U = (G;\alpha,\beta)
\]
where $G \in \OGraph_T$ is the skeleton and $\alpha,\beta$ assign
attachment objects in the appropriate fibres (of the forgetful functor
$\pi : \RMGState \to \OGraph_T$, see \cref{sec:determinism}) to nodes
and edges of $G$.
\end{definition}

This separates the global state into a base skeleton and recursively
attached subgraphs.

Rewriting operates in two ``planes'':

\begin{itemize}[leftmargin=*]
  \item \emph{attachment steps} are DPOI steps in the fibres
    $\alpha(v)$, $\beta(e)$ that do not change $G$;
  \item \emph{skeleton steps} are DPOI steps on $G$ itself, subject to
    rules ensuring that attachments at preserved positions can be
    transported.
\end{itemize}

\begin{definition}[Tick]\label{def:rmg-tick}
A \emph{tick} on an RMG state $U = (G;\alpha,\beta)$ consists
of a finite family of attachment steps in the fibres over $G$ followed
by a finite family of skeleton steps on $G$, chosen by the scheduler.
In \cref{sec:determinism} we impose additional conditions (independence,
scheduler--admissible batches, and the no-delete/no-clone-under-descent
invariant) on the ticks generated by the runtime.
\end{definition}

We now turn to the determinism properties of this semantics.
