\sectionbreak
\section{Rulial Distance: A Computable Quasi-Pseudometric on Observer Space}
\label{sec:rulial}

We next formalize observers and an MDL-based distance between them, the
\emph{rulial distance}.  This endows observer space with a computable
geometry on different descriptions of the same underlying RMG universe.

\subsection{Observers as functors}

Fix an RMG universe $\mathcal{U}$ together with a rule pack $R$ and its
history category $\Hist(\mathcal{U},R)$ whose objects are states
$U,V \in \mathcal{U}$ and whose morphisms are derivation paths between
them.  An \emph{observer} is a functor
\[
  O : \Hist(\mathcal{U},R) \To \mathcal{Y},
\]
where $\mathcal{Y}$ is a suitable category of observations (symbol
streams, trace graphs, etc.).  We assume that $O$ is realised by some
algorithm subject to fixed time and memory budgets $(\tau,m)$; these
budgets are reflected in the subscript of $D_{\tau,m}$ below.
Different observers may:

\begin{itemize}[leftmargin=*]
  \item choose different projections of the same wormhole payloads;
  \item aggregate or forget structure;
  \item expose different notions of causality.
\end{itemize}
Figure~\ref{fig:observer-projections} illustrates several such
projections.

\subsection{Translators, MDL Complexity, and Distortion}

A \emph{translator} between observers $O_1$ and $O_2$ is a functorial
construction
\[
  T_{12} : O_1 \Rightarrow O_2
\]
realised as a small DPOI transducer: for each history
$h \in \Hist(\mathcal{U},R)$ it maps the trace $O_1(h)$ to a trace
$T_{12}(O_1(h))$ in the observation category~$\mathcal{Y}$.  Likewise
we consider translators $T_{21} : O_2 \Rightarrow O_1$.

\paragraph{Example (SQL$\leftrightarrow$AST translator).}
Consider two observers of an RMG universe modeling a database
query planner.  Observer $O_1$ sees the wormhole payload $P$ as
a sequence of AST transformations (parse tree $\to$ optimized AST
$\to$ query plan), while observer $O_2$ sees only the initial SQL
string and final execution trace.  A translator $T_{12}$ must
reconstruct the SQL from the AST evolution: it can parse the initial
AST root, emit the corresponding SQL, and summarize the execution
steps by their side effects.  The reverse translator $T_{21}$ parses
the SQL and heuristically infers an AST evolution consistent with the
execution trace, incurring some distortion.  The description lengths
$\mathrm{DL}(T_{12}), \mathrm{DL}(T_{21})$ and distortion costs
quantify how ``close'' these two viewpoints are in rulial space.

Let $\mathrm{DL}(T)$ be a prefix-code description length for a
translator $T$ (its MDL cost).  Let
\[
  \mathrm{dist}_{\mathrm{tr}} : \mathcal{Y} \times \mathcal{Y}
    \to \mathbb{R}_{\ge 0}
\]
be a metric on individual traces (for example, an $L_1$ distance on
symbol streams or an edit distance on labelled paths).  We lift this
pointwise to observers
by defining, for observers $O,O' : \Hist(\mathcal{U},R) \to \mathcal{Y}$,
\[
  \mathrm{Dist}(O,O')
  := \sup_{h \in \Hist(\mathcal{U},R)}
       \mathrm{dist}_{\mathrm{tr}}\bigl(O(h),O'(h)\bigr).
\]
We assume all observers considered produce traces in a common metric
space of uniformly bounded diameter, so the supremum above is finite.
We also assume that post-composition by any translator is
$1$-Lipschitz:
\[
  \mathrm{Dist}(T\circ O, T\circ O')
  \le \mathrm{Dist}(O,O')
\]
for all translators $T$ and observers $O,O'$.

Fix a weighting parameter $\lambda>0$ that trades off description
length against distortion.

For time and memory budgets $(\tau,m)$ we write
$\Trans_{\tau,m}(O_1,O_2)$ for the set of translators from $O_1$ to
$O_2$ realisable within those budgets, and assume each budget class is
closed under finite composition.  We also assume a distinguished
identity translator $I_O : O \Rightarrow O$ for every observer $O$
with $\mathrm{DL}(I_O)=0$ and
$\mathrm{Dist}(O, I_O \circ O)=0$, corresponding to a null program
that simply re-emits its input.

We then define the budgeted MDL-based distance
\[
  D_{\tau,m}(O_1,O_2)
   := \inf_{\substack{
         T_{12}\in\Trans_{\tau,m}(O_1,O_2)\\
         T_{21}\in\Trans_{\tau,m}(O_2,O_1)}}
      \Bigl(
        \mathrm{DL}(T_{12}) + \mathrm{DL}(T_{21})
        + \lambda \bigl(
           \mathrm{Dist}(O_2, T_{12}\circ O_1) +
           \mathrm{Dist}(O_1, T_{21}\circ O_2)
        \bigr)
      \Bigr).
\]

\begin{theorem}[Basic properties of $D_{\tau,m}$]\label{thm:rulial-basic}
For all observers $O_1,O_2$ and budgets $(\tau,m)$, the distance
$D_{\tau,m}(O_1,O_2)$ is nonnegative and symmetric, and
$D_{\tau,m}(O,O)=0$ for every observer $O$.
\end{theorem}

\begin{proof}
Nonnegativity and symmetry are immediate from the definition of
$D_{\tau,m}$ as an infimum over sums of nonnegative symmetric terms.

For self-distance, consider the pair of identity translators
$(I_O,I_O)$.  By assumption $\mathrm{DL}(I_O)=0$ and the distortions
$\mathrm{Dist}(O,I_O\circ O)$ vanish, so the objective value of
$(I_O,I_O)$ is zero.  Hence $D_{\tau,m}(O,O)\le 0$, and
nonnegativity implies $D_{\tau,m}(O,O)=0$.
\end{proof}

\begin{theorem}[Main theorem on rulial distance (triangle inequality)]\label{thm:rulial-triangle}
Assume:
\begin{enumerate}[leftmargin=*]
  \item the description length $\mathrm{DL}$ is based on a prefix code
    and satisfies, for some constant $c\ge 0$,
    \[
      \mathrm{DL}(T_{13})
      \le \mathrm{DL}(T_{12}) + \mathrm{DL}(T_{23}) + c
    \]
    whenever $T_{13}$ is a composition $T_{23}\circ T_{12}$;
  \item the lifted distortion measure $\mathrm{Dist}$ is a metric on
    observers and post-composition by any translator is
    $1$-Lipschitz:
    $\mathrm{Dist}(T\circ O,T\circ O') \le \mathrm{Dist}(O,O')$;
  \item for each budget $(\tau,m)$ the classes
    $\Trans_{\tau,m}(O_i,O_j)$ are closed under finite composition.
\end{enumerate}
Then $D_{\tau,m}$ satisfies the triangle inequality up to an additive
constant:
\[
  D_{\tau,m}(O_1,O_3)
  \le D_{\tau,m}(O_1,O_2) + D_{\tau,m}(O_2,O_3) + 2c.
\]
In particular, together with \cref{thm:rulial-basic} this makes
$D_{\tau,m}$ a quasi-pseudometric (a pseudometric up to additive slack
$2c$) on observers.
\end{theorem}

\begin{proof}
Fix $\varepsilon>0$ and choose near-optimal translators
$(T_{12},T_{21})$ and $(T_{23},T_{32})$ attaining the infima for
$D_{\tau,m}(O_1,O_2)$ and $D_{\tau,m}(O_2,O_3)$ up to $\varepsilon/2$.
Form composite translators $T_{13}=T_{23}\circ T_{12}$ and
$T_{31}=T_{21}\circ T_{32}$.  By the subadditivity of $\mathrm{DL}$,
\[
  \mathrm{DL}(T_{13}) \le \mathrm{DL}(T_{12})+\mathrm{DL}(T_{23})+c,
  \qquad
  \mathrm{DL}(T_{31}) \le \mathrm{DL}(T_{21})+\mathrm{DL}(T_{32})+c.
\]
By the triangle inequality for $\mathrm{Dist}$ and the $1$-Lipschitz
property of post-composition, we have
\begin{align*}
  \mathrm{Dist}(O_3,T_{13}\circ O_1)
    &= \mathrm{Dist}(O_3,T_{23}\circ T_{12}\circ O_1)\\
    &\le \mathrm{Dist}(O_3,T_{23}\circ O_2)
        + \mathrm{Dist}(T_{23}\circ O_2, T_{23}\circ T_{12}\circ O_1)\\
    &\le \mathrm{Dist}(O_3,T_{23}\circ O_2)
        + \mathrm{Dist}(O_2, T_{12}\circ O_1),
\end{align*}
and similarly with roles reversed.

Summing these bounds and using the near-optimality of the chosen
translators yields
\[
  D_{\tau,m}(O_1,O_3)
  \le D_{\tau,m}(O_1,O_2) + D_{\tau,m}(O_2,O_3) + 2c + \varepsilon.
\]
Since $\varepsilon>0$ was arbitrary, the inequality without
$\varepsilon$ follows.
\end{proof}

The quantity $D_{\tau,m}$ is the \emph{rulial distance} between
observers: it measures how hard it is to translate between descriptions
of the same underlying history.  Observers with small distance live in
nearby ``frames''; those with large distance inhabit distant regions of
the Ruliad.

\subsection{Observer projections of wormholes}

Given a wormhole $(S_0,P)$, different observers may:

\begin{itemize}[leftmargin=*]
  \item expose only coarse-grained stages of $P$ (e.g.\ AST$\to$IR$\to$SQL);
  \item restrict to semantic effects (e.g.\ DB schema, invariants);
  \item highlight only adversarial branches;
  \item or inspect every microstep.
\end{itemize}

The holographic encoding thus supports a wide range of observer
perspectives from a single payload.

\begin{figure}[t]
  \centering
  \begin{tikzpicture}[
      wormhole/.style={rectangle,draw=green!60!black,fill=green!5,thick,rounded corners,
                       minimum width=32mm,minimum height=14mm,align=center},
      observer/.style={rectangle,draw=blue!70!black,fill=blue!8,thick,rounded corners=3pt,
                       minimum width=18mm,minimum height=8mm,align=center,font=\small},
      arrow/.style={-Latex,thick},
      >=Latex
    ]

    % Central wormhole
    \node[wormhole] (W) at (0,0)
      {wormhole\\[-1pt]
       \scriptsize $(S_0,P)$};

    % Observers
    \node[observer] (O1) at (-3.5,2.2) {$O_1$\\[-2pt]\scriptsize coarse stages};
    \node[observer] (O2) at (3.5,2.2) {$O_2$\\[-2pt]\scriptsize semantic};
    \node[observer] (O3) at (-3.5,-2.2) {$O_3$\\[-2pt]\scriptsize adversarial};
    \node[observer] (O4) at (3.5,-2.2) {$O_4$\\[-2pt]\scriptsize full microsteps};

    % Projections
    \draw[arrow,blue!70!black] (W.north west) -- (O1.south east);
    \draw[arrow,blue!70!black] (W.north east) -- (O2.south west);
    \draw[arrow,blue!70!black] (W.south west) -- (O3.north east);
    \draw[arrow,blue!70!black] (W.south east) -- (O4.north west);

    % Labels on arrows
    \node[rotate=45,font=\scriptsize] at (-1.8,1.2) {project};
    \node[rotate=-45,font=\scriptsize] at (1.8,1.2) {project};
    \node[rotate=-45,font=\scriptsize] at (-1.8,-1.2) {project};
    \node[rotate=45,font=\scriptsize] at (1.8,-1.2) {project};

  \end{tikzpicture}
  \caption{Multiple observers projecting the same wormhole $(S_0,P)$
  into different trace formats.  Each observer $O_i$ extracts a
  different view of the interior evolution: coarse-grained stages,
  semantic invariants, adversarial branches, or full microsteps.
  The rulial distance measures the complexity of translating
  between these views.}
  \label{fig:observer-projections}
\end{figure}
