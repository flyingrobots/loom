\sectionbreak
\section{Discussion and Future Work}
\label{sec:discussion}

We have defined Recursive Metagraphs, given a deterministic concurrent
DPOI semantics, proved key confluence properties, and introduced a
holographic provenance model in which the boundary of a computation
encodes its entire interior evolution.  We have also outlined an
MDL-based rulial geometry on observers and connected RMG rewriting to
multiway systems.

\subsection{Implementation guarantees (Echo)}

In the concrete \AION{} runtime (Echo), the semantic assumptions above
are enforced by operational determinism invariants; any violation aborts
the tick deterministically and emits an error node for replay analysis.
These invariants are also exercised by the test suite to guarantee
bit-level reproducibility:
\begin{itemize}[leftmargin=*]
  \item \textbf{World Equivalence:} identical diff sequences and merge
    decisions yield identical world hashes.
  \item \textbf{Merge Determinism:} given the same base snapshot, diffs,
    and merge strategies, the resulting snapshot and diff hashes are
    identical.
  \item \textbf{Temporal Stability:} GC, compression, and inspector
    activity do not alter logical state.
  \item \textbf{Schema Consistency:} component layout hashes must match
    before merges; mismatches block the merge.
  \item \textbf{Causal Integrity:} writes cannot modify values they
    transitively read earlier in Chronos; paradoxes are detected and
    isolated.
  \item \textbf{Entropy Reproducibility:} branch entropy is a
    deterministic function of recorded events.
  \item \textbf{Replay Integrity:} replaying from node $A$ to $B$
    produces identical world hash, event order, and PRNG draw counts.
\end{itemize}

\subsection{Related work}

Our work builds on several research traditions:

\paragraph{Algebraic graph rewriting.}
The DPO (Double Pushout) approach to graph rewriting was introduced
by Ehrig and others~\cite{EhrigLowe1997,EEPT06} and extended to
adhesive categories by Lack and Soboci{\'n}ski~\cite{LS06}.  We build
directly on these foundations, applying DPO semantics both to the
skeleton plane and to attachment fibers.  The tick-level confluence
theorem (\cref{thm:tick-confluence}) is a specialization of the
standard concurrency theorem for adhesive systems.

Category-theoretic graph rewriting has also been applied to
discretised space--time models in
Arrighi--Costes--Maignan~\cite{Arrighi2025Reversible}, which uses DPO
rewriting in an adhesive setting to study space--time reversible graph
rewriting.  Our use of the same machinery is conceptually similar, but
we focus on deterministic multiway semantics
and holographic provenance in a computational setting rather than on
physical geometry.

\paragraph{Confluence and termination.}
The critical-pair lemma and Newman's lemma are classical tools in
term rewriting; for decreasing-diagram techniques, see van
Oostrom~\cite{vanOostrom1994}.  Our conditional global confluence
result (\cref{thm:global}) invokes these standard methods in
the graph-rewriting setting.

\paragraph{Multiway systems and the Ruliad.}
Wolfram~\cite{Wolfram2020} introduced multiway systems and the
Ruliad as a framework for fundamental physics and metamathematics.
Our RMG rewriting naturally induces multiway graphs; the determinism
discipline we impose selects unique worldlines within this larger
possibility space.  The rulial distance in Section~\ref{sec:rulial}
is our contribution to the problem of quantifying observer
differences.

\paragraph{Minimum Description Length.}
The MDL principle, pioneered by Rissanen~\cite{Rissanen1978}, provides
a rigorous information-theoretic basis for model selection and
compression.  We apply MDL to measure the complexity of
observer-to-observer translators, yielding a computable
quasi-pseudometric on the space of descriptions.

\paragraph{Categorical computation and diagrammatic reasoning.}
String diagrams and categorical algebra have been successfully applied
to quantum computing and concurrency~\cite{CoeckeDuncan2011}.  Our
two-plane fibration view (\cref{sec:determinism}) is in this
spirit: attachments live in fibers, and reindexing functors transport
attachment updates along skeleton morphisms.

The novelty of our approach lies not in any single component, but in
the synthesis: combining DPO rewriting on recursive structures,
deterministic concurrency via a two-plane discipline, and holographic
provenance encoding, all within a single framework with explicit
confluence guarantees.

Several directions remain:

\begin{itemize}[leftmargin=*]
  \item \textbf{Full global confluence analysis.}  For concrete rule
    packs used in practice, automated critical-pair analysis and
    decreasing-diagram labellings can provide machine-checkable
    confluence certificates.
  \item \textbf{Zero-knowledge provenance.}  Because payloads identify
    substructures via opaque, content-addressed pointers, it is natural
    to layer cryptographic commitments and zero-knowledge proofs on top,
    enabling external verifiers to check correctness properties without
    learning private data.
  \item \textbf{Temporal logic and the Time Cone.}  The Chronos, Kairos, and
    Aion triad naturally suggests new modal and temporal logics for
    reasoning about linear time, branch points, and the surrounding
    possibility space.
  \item \textbf{\COMPUTER{} architecture.}  Building on this foundation,
    the companion paper will define the \AION{} \COMPUTER{}: a machine model
    whose basic step is a provenance-carrying RMG rewrite, supporting
    backward- and forward-traceable computation, multiverse debugging,
    and glass-box AI cognition.
\end{itemize}

The long-term vision is that computational holography becomes as
standard as content-addressing and version control are today: every
nontrivial system records not just \emph{what} happened, but a
compact, verifiable encoding of \emph{how} it happened.
